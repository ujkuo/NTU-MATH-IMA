\documentclass[a4paper]{article}
%The main tex begins from line 262
\usepackage[top = 2.5cm, bottom = 2.5cm, left = 2cm, right = 2cm]{geometry} 
\usepackage{amsmath} % For the usage of \because and \therefore
\usepackage{amssymb}
\usepackage{fancyhdr}
\usepackage{lastpage}
\usepackage{etoolbox}
\usepackage{indentfirst}
\usepackage{tabularx}
\usepackage{mathtools}
\usepackage{booktabs}
\usepackage{authblk}
\usepackage[calcwidth]{titlesec}
\usepackage{bookmark}
\usepackage[capitalize]{cleveref}
\usepackage{array}
\usepackage[english]{babel}
\usepackage{enumitem} % For the usage of labeling in enumerate and itemize
%\usepackage[utf8]{inputenc}
%\usepackage{CJKutf8}
\usepackage{xeCJK}

%%%%%%%%%%%%%%%%%%%%%%%%%%%%%%%%%%%%%%%%%%%%%%%%%%%

\makeatletter
% Macros for setting basic info
\gdef\@uni{National Taiwan University}
\gdef\@department{Information Management}
\gdef\@me{Yu-Chieh Kuo}

\newcommand{\course}[2][]{
    \ifstrempty{#1}{
        \gdef\shortcourse{#2}
    }{
        \gdef\shortcourse{#1}
    }
    \gdef\@course{#2}
}
\newcommand{\teacher}[1]{\gdef\@teacher{#1}}
\newcommand{\semester}[1]{\gdef\@semester{#1}}
\newcommand{\uni}[1]{\gdef\@uni{#1}}
\newcommand{\department}[1]{\gdef\@department{#1}}
\newcommand{\student}[3][]{
    \ifstrempty{#1}{
        \author{\textbf{#2} \textbf{#3}}
    }{
        \author[#1]{\textbf{#2} \textbf{#3}}
    }
}
\newcommand{\assignment}[2][Homework]{
    \ifstrempty{#2}{
        \gdef\@assignment{#1}
    }{
        \gdef\@assignment{#1 #2}
    }
}

% page style for title page
\fancypagestyle{title}{
    \fancyhf{}
    \renewcommand{\headrulewidth}{0pt}
    \cfoot{\footnotesize Page {\thepage} of \pageref{LastPage}} 
}
\newcommand{\ujtitle}{
    \thispagestyle{title}
    \noindent\begin{tabularx}{\linewidth}{Xr} % This is a simple tabular environment to align your text nicely 
    {\large \bf{\@course}} \\
    National Taiwan University, \@semester  \\
    \bottomrule % \hline produces horizontal lines.
    \end{tabularx}
    
    \vspace*{0.3cm}

    \begin{center}
        {\huge\bf\textsc{\@assignment}}\\[3mm]
        {\@author}
    \end{center}

    \vspace*{0.3cm}
}
\newcommand{\mymaketitle}{
    \thispagestyle{title}
    \vspace*{-15mm}
    \noindent\begin{tabularx}{\linewidth}{Xr}
                                & \textsl{National Taiwan University} \\
        \large\textbf{\@semester,} & \textsl{Department of \@department}         \\
        \large\textbf{\@course} & \textsl{Prof. \@teacher}\Bstrut\\
        \bottomrule
    \end{tabularx}

    \vspace*{10mm}

    \begin{center}
        {\huge\textsc{\@assignment{}}}\\[6mm]
        {\@author}
    \end{center}

    \vspace*{1cm}
}

% page style for contents
\pagestyle{fancy}
\fancyhf{}
\setlength{\headheight}{13.6pt}
\lhead{\small {\@course}: {\@assignment}}
\rhead{\small {\@me}}
\cfoot{\small {Page {\thepage} of \pageref{LastPage}}} 

% Style for part, section, and subsection
% Part
\titleclass{\part}{straight}
\titleformat{\part}
[display]
{\centering\bfseries}
{\setlength{\titlewidth}{1.5\titlewidth}\large\titleline*[c]{\titlerule[.6pt]}\vspace{3pt}\MakeUppercase{\partname} \thepart}
{0pt}
{\LARGE\itshape}
[{
    \setlength{\titlewidth}{1.5\titlewidth}
    \titleline*[c]{\titlerule*{\mdseries-}}
}]
\titlespacing*{\part}{0pt}{20pt}{20pt}[0pt]

% Problem and subproblem
\setcounter{secnumdepth}{0}
\newcounter{problem}
\newcounter{subproblem}[problem]
\newbool{appendix}
\gdef\problemName{Problem}
\newcommand{\problem}[2][]{
    \setcounter{equation}{0}
    \ifstrempty{#1}{
        \stepcounter{problem}
    }{
        \setcounter{problem}{#1}
    }
    \ifstrempty{#2}{
        \ifbool{appendix}{
            \section{\hspace{-5mm}\problemName{} \Alph{problem}.}
        }{
            \section{\hspace{-5mm}\problemName{} \arabic{problem}}
        }
    }{
        \ifbool{appendix}{
            \section{\hspace{-5mm}\problemName{} \Alpa{problem}: #2}
        }{
            \section{\hspace{-5mm}\problemName{} \arabic{problem}: #2}
        }
    }
}
\newcommand{\subproblem}[2][]{
    \setcounter{equation}{0}
    \ifstrempty{#1}{
        \stepcounter{subproblem}
    }{
        \setcounter{subproblem}{#1}
    }
    \ifstrempty{#2}{
        \ifbool{appendix}{
            \section{\hspace{-2mm}\Alph{problem}.\arabic{subproblem}}
        }{
            \section{\hspace{-2mm}\arabic{problem}.(\alph{subproblem})}
        }
    }{
        \ifbool{appendix}{
            \section{\hspace{-2mm}\Alph{problem}.\arabic{subproblem}. #2}
        }{
            \section{\hspace{-2mm}\arabic{problem}.(\alph{subproblem}) #2}
        }
    }
}

% Appendix
\newcommand{\myAppendix}{
    \setcounter{section}{0}
    \renewcommand{\sectionname}{Appendix}
    \renewcommand{\thesection}{\Alph{section}}
    \renewcommand{\thesubsection}{\Alph{section}.\arabic{subsection}}
    \renewcommand{\perhapscolon}[1]{\perhapsdot{##1}}
}

% Bookmarks
\hypersetup{
    bookmarksnumbered=true
}

% Reference nome for task and subtask
\crefname{section}{\sectionname}{\sectionnamepl}

\makeatother

%%%%%%%%%%%%%%%%%%%%%%%%%%%%%%%%%%%%%%%%%%%%%%%%%%%

% \everymath{\displaystyle} % display all math symbol as DISPLAY MODE

% commands for spacing
\newcommand{\Tstrut}{\rule{0pt}{4mm}}         % = `top' strut
\newcommand{\Bstrut}{\rule[-2mm]{0mm}{0mm}}   % = `bottom' strut

% Paired Delimiters {}, (), []
\providecommand\given{} % so it exists
\newcommand\SetSymbol[1][]{
   \nonscript\,#1\vert \allowbreak \nonscript\,\mathopen{}}
\DeclarePairedDelimiterX\Set[1]{\lbrace}{\rbrace}%
 { \renewcommand\given{\SetSymbol[\delimsize]} #1 }
\DeclarePairedDelimiterX{\Bkt}[1]{[}{]}{#1}
\DeclarePairedDelimiterX{\Paren}[1]{(}{)}{#1}
\DeclarePairedDelimiterX{\Abs}[1]{\lvert}{\rvert}{#1}
\newcommand{\PR}[1]{\Pr\Bkt*{#1}}
\newcommand{\overbar}[1]{\mkern 1.5mu\overline{\mkern-1.5mu#1\mkern-1.5mu}\mkern 1.5mu}

\newcolumntype{D}{>{$(}l<{)$}}
\newcolumntype{R}{>{\displaystyle}r}
\newcolumntype{C}{>{\displaystyle}c}
\newcolumntype{L}{>{\displaystyle}l}

%%%%%%%%%%%%%%%%%%%%%%%%%%%%%%%%%%%%%%%%%%%%%%%%%%%

\usepackage{amsfonts}
\usepackage{amsthm}

%%%%%%%%%%%%%%%%%%%%%%%%%%%%%%%%%%%%%%%%%%%%%%%%%%%

% Declare operator and useful math command
\DeclareMathOperator{\EOp}{\mathbb{E}}
\newcommand{\E}[1]{\ensuremath{\EOp\Bkt*{#1}}}
\newcommand{\R}{\mathbb{R}}
\newcommand{\N}{\mathbb{N}}
\newcommand{\Q}{\mathbb{Q}}
\newcommand{\Z}{\mathbb{Z}}
\newcommand{\PS}[1]{\mathcal{P}(#1)}
\newcommand{\ve}{\varepsilon}
\newcommand{\es}{\emptyset}

\newcommand{\sst}{\subset}
\newcommand{\sse}{\subseteq}
\newcommand{\spt}{\supset}
\newcommand{\spe}{\supseteq}

\newcommand{\ie}{\text{ i.e., }}
\newcommand{\st}{\text{ s.t.  }}

% Declare delimiter
\DeclareMathDelimiter{\lVert}
  {\mathopen}{symbols}{"6B}{largesymbols}{"0D}
\DeclareMathDelimiter{\rVert}
  {\mathclose}{symbols}{"6B}{largesymbols}{"0D}
\DeclarePairedDelimiter\norm{\lVert}{\rVert}
\DeclarePairedDelimiter\nrm{\lvert}{\rvert}
\DeclarePairedDelimiter\vct{\langle}{\rangle}

%%%%%%%%%%%%%%%%%%%%%%%%%%%%%%%%%%%%%%%%%%%%%%%%%%%

\theoremstyle{plain}
\newtheorem{corollary}{Corollary}
\newtheorem{proposition}{Proposition}

%%%%%%%%%%%%%%%%%%%%%%%%%%%%%%%%%%%%%%%%%%%%%%%%%%%

\setCJKmainfont{思源宋體 TW}
%\setmainfont{思源宋體 TW}

%%%%%%%%%%%%%%%%%%%%%%%%%%%%%%%%%%%%%%%%%%%%%%%%%%%

% Use listing package to display different programming language
% Use xcolor package to display syntax color for programming language
% Usage: \env{lstlisting}[language=LANGUAGENAME]
% Common language name: Python, bash, C, C++, R, sh, make, Matlab

\usepackage{listings}
\usepackage{xcolor}

\definecolor{dkgreen}{rgb}{0,0.6,0}
\definecolor{gray}{rgb}{0.5,0.5,0.5}
\definecolor{mauve}{rgb}{0.58,0,0.82}
\lstset{frame=tb,
  language=Python,
      aboveskip=3mm,
    belowskip=3mm,
    showstringspaces=false,
  columns=flexible,
      basicstyle={\small\ttfamily},
    numbers=none,
    numberstyle=\tiny\color{gray},
  keywordstyle=\color{blue},
      commentstyle=\color{dkgreen},
    stringstyle=\color{mauve},
    breaklines=true,
  breakatwhitespace=true,
      tabsize=3
  }

%%%%%%%%%%%%%%%%%%%%%%%%%%%%%%%%%%%%%%%%%%%%%%%%%%%

\begin{document}
%\begin{CJK*}{UTF8}{bsmi}
\course{Introduction to Mathematical Analysis (I)}
\semester{Fall 2021}
%\teacher{}
\department{}
\assignment{4}
\date{\today}
\student[$\dagger$]{Yu-Chieh Kuo}{B07611039}
\affil[$\dagger$]{Department of Information Management, National Taiwan University}
\ujtitle
\section{Notations used in the homework}
To avoid notations confusion, we define the common used notations initially.
\begin{enumerate}
    \item Define $N_r(x)$ as the neighborhood centering at $x$ in radius $r$.
        On the other hand, $N(x)$ indicates the neighborhood centering at $x$ without
        specific radius.
    \item Define $B_r(x)$ as the open ball centering at $x$ in radius $r$.
\end{enumerate}

%% p1
\problem{}
Following the {\it hint}, we fix $\delta > 0$ and pick $x_1$ in $X$,
then choose $x_{j+1} \st d(x_i, x_{j+1}) \geq \delta$ for $i=1,\dots,j$.
Let $E_\delta \sse X$ be the set collecting these $x_i$.
Next, we need to show that this process must stop after finite number of steps and that
$X$ can therefore be covered by finite many neighborhoods of radius $\delta$.
This statement can be proved by contradiction. Assume by contradiction,
the process will not stop after finite number of steps i.e., $E_\delta$ is an
infinite subset of $X$, and $E_\delta$ has a limit point $p \in X$ since it states
that all infinite subsets of $X$ have a limit point.
Moreover, based on {\bf Theorem 2.20} in Rudin
\footnote{If $p$ is a limit point of a set $E$, then every neighborhood of $p$
contains {\it infinitely} many points of $E$.}
, we define two points $q, r$ within $N_{\frac{\delta}{4}}(p)$. 
Note that $q,r \in N_{\frac{\delta}{4}}(p) \sse E_\delta$.
Therefore, by triangle inequality, we have
\[
    d(q,r) \leq d_(q,p) + d(p,r) < \frac{\delta}{4} +  \frac{\delta}{4} < \delta,
\]
which contradicts the construction of $E_\delta$.
Consequently, it's proved that $X$ can be covered by finite many neighborhoods of 
radius $\delta$.

By the definition of {\it separable}, $X$ is separable if it has a countable
dense subset. We derive that it has a countable subset, then we are going to
find its dense set. The definition of dense set $D \sse X$ is
\[
    \forall \ve>0 \, \forall x \in X, \, D \cap B_\ve(x) \neq \emptyset,
\]
and, by the Archimedean property, the definition can be re-write to
\[
    \forall n\in\N \, \forall x \in X, \, D \cap B_{\frac{1}{n}}(x) \neq \emptyset.
\]
Use $E_\delta$ again. Take $\delta = \frac{1}{n}$, and define the union of all \
$E_\delta$ as 
\[
    E = \bigcap_{n=1}^\infty E_\frac{1}{n} \sse X.
\]
Here $E$ is dense in $X$, since given any $p \in X-E$, there exists 
$q \in B_r(p)$, where $q \in E$. Pick any $n \in \Z_+ \st \frac{1}{n} < r$,
it suffices to say that $q \in B_r(p)$, that is, $E$ is dense in $X$.

Finally, since there is a countable dense subset of $X$, we could say $X$
is separable.

% p2
\problem{}
Since metric space $K$ is compact, each of $K$'s open covers has a finite subcover.
In other words, for every collection $S$ of open subsets of $K$,
$K \sse \bigcup_{x\in S}x$, and there is a finite subset $S'$ of $S$ such that
$K \sse \bigcup_{x\in S'}x$.

Given any $\delta > 0$, $\delta \in \R$, define an open cover $C$ of $K$ by
$C = \{B_\delta(p): p \in K\}$, and its finite subcover $F$ of $K$.
Here the union of $F$ can be the finite collection of $B_\delta(p): p \in K$,
and $K \sse \bigcup_{x\in F}x$. It's equivalent to $K$ is bounded, then,
by {\bf Problem 1}, we can say that $K$ is separable.

% p3
\problem{}
By the results in {\bf Problem 1} and {\bf Ex. 2.23} in Rudin (one of the problems
in Homework 3), we have that $X$ is separable and countable.
Let $V = \{V_n\}$ be a countable base of $X$. Give any open cover
$C$ of $X$, for all $V_n \in V$, we take $G_n \in C$ containing $V_n$ and collect
all $G_n$ as $G$, i.e.,
\[
    G = \{G_n : B_n \sse G_n \forall n \in \Z_+\}. 
\]
Note that $G$ is a countable subset of $C$ and $G$ covers $X$ since $V$ is a 
countable base of $X$. Hence, given any open cover $C$ of $X$, there is 
a countable subcover $G$ of $X$. Next, assume by contradiction, if
there were no finite subcover of $G$ covering $X$, then the complement
$F_n$ of $G_1\cup\cdot\cup G_n$ is nonempty for each $n$, but $\cap F_n$ is empty.
Let $E$ be a set containing a point from each $F_n$, $E$ here is infinite and has a
limit point $p$. Note that $p \in G_n$ for some $n$ since $G$ is an open cover
of $X$. In addition, since $G_n$ is open, there exists an open neighborhood $N(p)$
such that $N(p) \sse G_n$. By the construction of $F_n$,
$N(p) \cap F_n = \es$, contrary to the assumption that $p$ is a limit point of $E$.
Consequently, $X$ is compact.


% p4
\problem{}
\subproblem{} %p4-1
Let $p^{-1} = \alpha, q^{-1} = 1-\alpha$ and re-write the right-hand-side to
\[
    \alpha u^p + (1-\alpha)v^q.
\]
Take logarithm to this function, it becomes
\[
    \ln(\alpha u^p + (1-\alpha)v^q),
\]
and due to the concavity of logarithm, we have the inequality as
\[
    \begin{array}{RCL}
        \ln(\alpha u^p + (1-\alpha)v^q) & \geq
        & \alpha\ln(u^p) + (1-\alpha)\ln(v^q) \\
        & = & \alpha p\ln(u) + (1-\alpha)q\ln(v) \\
        & = & \ln(u) + \ln(v).
    \end{array}
\]
This inequality is equivalent to $uv \leq \frac{u^p}{p} + \frac{v^q}{q}$.
In addition, by the proposition of concavity, the equality holds if 
and only if $u^p = v^q$.
    
\subproblem{} % p4-2
We can trickily focus on the case $\nrm{x}_p = \nrm{y}_p = 1$, as all positive numbers
are scalars for 1, and multiplying scalars on both sides in the equality is linear
i.e., 
\[
    \nrm{\vct{ax,by}} \leq \nrm{ax}_p\nrm{by}_p
\]
holds for arbitrary $a,b$.
Starting to our derivation.
\[
    \begin{array}{RCL}
        \nrm{\vct{x,y}} & = & \nrm{x_1y_1 + x_2y_2 + \dots + x_ky_k} \\ [2mm]
        & \leq & \nrm{x_1}\nrm{y_1} + \nrm{x_2}\nrm{y_2} +\dots + \nrm{x_k}\nrm{y_k}
        \quad \text{(By triangle inequality.)} \\ [2mm]
        & \leq & \left(\frac{\nrm{x_1}^p}{p} + \frac{\nrm{y_1}^q}{q}
        + \dots + \frac{\nrm{x_k}^p}{p} + \frac{\nrm{y_k}^q}{q}\right) 
        \quad \text{(By {\bf 4(a)}).} \\
        & = & \frac{1}{p}\sum_{i=1}^k\nrm{x_i}^p + \frac{1}{q}\sum_{i=1}^k\nrm{y_i}^p
        \quad \text{(Note that $(\nrm{x}_p)^p = (\nrm{y}_p)^p = 1$ as well.)} \\
        & = & \frac{1}{p} + \frac{1}{q} \\ [2mm]
        & = & 1.
    \end{array}
\]
Here we prove that $\nrm{\vct{x,y}} \leq \nrm{x}_p\nrm{y}_p$ for the case
$\nrm{x}_p = \nrm{y}_p = 1$, and the inequality holds for any case as previous 
discussion.
The equality holds if and only if $\nrm{x}^p = \nrm{y}^q$.

\subproblem{} % p4-3
Starting from the case $\sum_{i=1}^k\nrm{x_i+y_i}^p$, we have
\[
    \begin{array}{RCL}
        \sum_{i=1}^k\nrm{x_i+y_i}^p & \leq & 
        \sum_{i=1}^k(\nrm{x_i}\cdot\nrm{x_i+y_i}^{p-1}
        + \nrm{y_i}\cdot\nrm{x_i+y_i}^{p-1})
        \quad \text{(By triangle inequality.)} \\
        & = & \sum_{i=1}^k\nrm{x_i}\cdot\nrm{x_i+y_i}^{p-1}
        + \sum_{i=1}^k\nrm{y_i}\cdot\nrm{x_i+y_i}^{p-1} \\
        & \leq & \left(\sum_{i=1}^k\nrm{x_i}^p\right)^{\frac{1}{p}}
        \left(\sum_{i=1}^k(\nrm{x_i+y_i}^{p-1})^q\right)^{\frac{1}{q}} \\ [6mm]
        & & + \left(\sum_{i=1}^k\nrm{y_i}^p\right)^{\frac{1}{p}}
        \left(\sum_{i=1}^k(\nrm{x_i+y_i}^{p-1})^q\right)^{\frac{1}{q}}
        \quad \text{(By {\bf 4(b).})} \\ [6mm]
        & = & \nrm{x}_p\left(\sum_{i=1}^k(\nrm{x_i+y_i}^p)\right)^{\frac{1}{q}}
        + \nrm{y}_p\left(\sum_{i=1}^k(\nrm{x_i+y_i}^p)\right)^{\frac{1}{q}}.
    \end{array}
\]
Move $\left(\sum_{i=1}^k(\nrm{x_i+y_i}^p)\right)^{\frac{1}{q}}$ to the left-hand-side,
the inequality becomes
\[
    \begin{array}{RL}
        & \sum_{i=1}^k\nrm{x_i + y_i}^p 
        \left(\sum_{i=1}^k(\nrm{x_i+y_i}^p)\right)^{\frac{-1}{q}}
        \leq \nrm{x}_p + \nrm{y}_p \\
        \iff & \left(\sum_{i=1}^k(\nrm{x_i+y_i}^p)\right)^{\frac{1}{p}}
        \leq \nrm{x}_p + \nrm{y}_p \\
        \iff & \nrm{x+y}_p \leq \nrm{x}_p + \nrm{y}_p.
    \end{array}
\]
The equality holds if and only if $\nrm{x+y}_p = \nrm{x}_p + \nrm{y}_p$.

\subproblem{} % p4-4
Since
\[
    \nrm{x}_p = \left(\sum_{i=1}^k\nrm{x}^p \right)^\frac{1}{p}
    \leq \left(\sum_{i=1}^k \max_i \nrm{x_i}^p\right)^\frac{1}{p}
    = k^\frac{1}{p} \max_{1\leq i\leq k}\nrm{x_i},
\]
when $p \to \infty$, $\lim_{p\to\infty}\nrm{x}_p \leq \max_{1\leq i\leq k}\nrm{x_i}$.
On the other hand,
\[
    \nrm{x}_p = \left(\sum_{i=1}^k\nrm{x_i}^p\right)^\frac{1}{p}
    \geq \left(\max_{1\leq i\leq k} \nrm{x_i}^p\right)^\frac{1}{p}
    = \max_{1\leq i\leq k} \nrm{x_i}.
\]
As $p \to \infty$, $\lim_{p\to\infty}\nrm{x}_p \geq \max_{1\leq i\leq k}\nrm{x_i}$.
Hence, we can state that $\lim_{p\to\infty}\nrm{x}_p = \max_{1\leq i\leq k}\nrm{x_i}$.

To establish inequalities among $l$-norm for $l = 1,p,q,\infty, \, 1<p<\infty$,
we compare the case $l=1,p,\infty$ initially.
It's clearly to say that $\nrm{x}_1 > \nrm{x}_p$, as 
$(\nrm{x}_1)^p \geq (\nrm{x}_p)^p$. It's also easy to observe that 
$\nrm{x}_p > \nrm{x}_\infty$, since
\[
    \nrm{x}_p = \left(\sum_{i=1}^k\nrm{x_i}^p\right)^\frac{1}{p}
    \geq \left(\max_{1\leq i\leq k}\nrm{x_i}^p\right)^\frac{1}{p}
    = \max_{1\leq i\leq k}\nrm{x_i} = \nrm{x}_\infty.
\]
Last, to compare the relation between $p$-norm and $q$-norm,
it's a good attempt to differentiate it and then observe the relation.
We have
\[
    \frac{\partial}{\partial {p}}\left(\sum_{i=1}^k\nrm{x_i}^p\right)^\frac{1}{p}
    = \left(\frac{-\ln(\sum_{i=1}^k x_i^p)}{p^2}
    + \frac{\sum_{i=1}^k x_i^p\ln(x_i)}{p\sum_{i=1}^k x_i^p}\right)
    \left(\sum_{i=1}^kx_i^p\right)^\frac{1}{p} < 0.
\]
Hence, as $p$ increases, the value of $p$-norm decreases. At the end,
we have the relation inequality that
$\nrm{x}_1 > \nrm{x}_p > \nrm{x}_q > \nrm{x}_\infty$, where $1 < p < q < \infty$.

\subproblem{} % p4-5
Yes, they have the same topology.

\subproblem{} % p4-6
No. Consider a counter-example $x=(0,1), y=(1,0)$ in $\R^2$. For $p \in (0,1),
\nrm{x+y}_p = 2^\frac{1}{p} > 2 = \nrm{x}_p + \nrm{y}_p$.


% p5
\problem{}
First we check $X$ is compact.
Let $O = \{O_i\}_{i \in S}$ be some open cover of $X$, $S$ be some proper set.
Assume that it has no finite subcover, for each finite $F \sse S$, there exists
some point $p \in X \st p \notin \bigcup_{i\in F}O_i$. This statement is
equivalent to that for each finite $F \sse S$, there exists some point
$p \in \bigcap_{i\in S}(X\setminus O_i)$. Hence, the collection 
$\{X\setminus O_i\}_{i\in S}$ is a collection of closed sets in $X$ having
the finite intersection property, which implies there is some point
$q \in \bigcap_{i\in S}(X\setminus O_i) \ie q \notin \bigcap_{i\in S}O_i$,
contrary to $\{O_i\}_{i\in S}$ is a cover of $X$. Hence, we conclude that
$\{O_i\}_{i\in S}$ has a finite subcover, i.e., for every collection of 
closed set in $X$, it has the finite intersection property. So $X$ is compact.

Next we check whether the property holds if $X$ is compact.
Assume that $X$ is compact and let $\{C_i\}_{i\in S}$ be some family of closed
subsets in $X$ having the finite intersection property.
Assume by contradiction that $\bigcap_{i\in S}C_i=\emptyset$, then
$\{X\setminus C_i\}_{i\in S}$ is an open cover of $X$, and it has a finite
subcover i.e., $\bigcup_{i\in F}(X\setminus C_i)=X$ for some finite subset
$F \sse S$. However, by asuumption it indicates $\bigcap_{i\in S}C_i=\emptyset$,
which contradicts that $\{C_i\}_{i\in S}$ has the finite intersection property.
Therefore we conclude that $\bigcap_{i\in S}C_i\neq\emptyset$.

%\end{CJK*}
\end{document}

