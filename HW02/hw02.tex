\documentclass[a4paper]{article}
%The main tex begins from line 188
\usepackage[top = 2.5cm, bottom = 2.5cm, left = 2cm, right = 2cm]{geometry} 
\usepackage{amsmath}
\usepackage{amssymb}
\usepackage{fancyhdr}
\usepackage{lastpage}
\usepackage{etoolbox}
\usepackage{indentfirst}
\usepackage{tabularx}
\usepackage{mathtools}
\usepackage{booktabs}
\usepackage{authblk}
\usepackage[calcwidth]{titlesec}
\usepackage{bookmark}
\usepackage[capitalize]{cleveref}
\usepackage{array}
\usepackage[english]{babel}
\usepackage{enumitem}
%\usepackage[utf8]{inputenc}
%\usepackage{CJKutf8}
\usepackage{xeCJK}

%%%%%%%%%%%%%%%%%%%%%%%%%%%%%%%%%%%%%%%%%%%%%%%%%%%

\makeatletter
% Macros for setting basic info
\gdef\@uni{National Taiwan University}
\gdef\@department{Information Management}
\gdef\@me{Yu-Chieh Kuo}

\newcommand{\course}[2][]{
    \ifstrempty{#1}{
        \gdef\shortcourse{#2}
    }{
        \gdef\shortcourse{#1}
    }
    \gdef\@course{#2}
}
\newcommand{\teacher}[1]{\gdef\@teacher{#1}}
\newcommand{\semester}[1]{\gdef\@semester{#1}}
\newcommand{\uni}[1]{\gdef\@uni{#1}}
\newcommand{\department}[1]{\gdef\@department{#1}}
\newcommand{\student}[3][]{
    \ifstrempty{#1}{
        \author{\textbf{#2} \textbf{#3}}
    }{
        \author[#1]{\textbf{#2} \textbf{#3}}
    }
}
\newcommand{\assignment}[2][Homework]{
    \ifstrempty{#2}{
        \gdef\@assignment{#1}
    }{
        \gdef\@assignment{#1 #2}
    }
}

% page style for title page
\fancypagestyle{title}{
    \fancyhf{}
    \renewcommand{\headrulewidth}{0pt}
    \cfoot{\footnotesize Page {\thepage} of \pageref{LastPage}} 
}
\newcommand{\ujtitle}{
    \thispagestyle{title}
    \noindent\begin{tabularx}{\linewidth}{Xr} % This is a simple tabular environment to align your text nicely 
    {\large \bf{\@course}} \\
    National Taiwan University, \@semester  \\
    \bottomrule % \hline produces horizontal lines.
    \end{tabularx}
    
    \vspace*{0.3cm}

    \begin{center}
        {\huge\bf\textsc{\@assignment}}\\[3mm]
        {\@author}
    \end{center}

    \vspace*{0.3cm}
}
\newcommand{\mymaketitle}{
    \thispagestyle{title}
    \vspace*{-15mm}
    \noindent\begin{tabularx}{\linewidth}{Xr}
                                & \textsl{National Taiwan University} \\
        \large\textbf{\@semester,} & \textsl{Department of \@department}         \\
        \large\textbf{\@course} & \textsl{Prof. \@teacher}\Bstrut\\
        \bottomrule
    \end{tabularx}

    \vspace*{10mm}

    \begin{center}
        {\huge\textsc{\@assignment{}}}\\[6mm]
        {\@author}
    \end{center}

    \vspace*{1cm}
}

% page style for contents
\pagestyle{fancy}
\fancyhf{}
\setlength{\headheight}{13.6pt}
\lhead{\small {\@course}: {\@assignment}}
\rhead{\small {\@me}}
\cfoot{\small {Page {\thepage} of \pageref{LastPage}}} 

% Style for part, section, and subsection
% Part
\titleclass{\part}{straight}
\titleformat{\part}
[display]
{\centering\bfseries}
{\setlength{\titlewidth}{1.5\titlewidth}\large\titleline*[c]{\titlerule[.6pt]}\vspace{3pt}\MakeUppercase{\partname} \thepart}
{0pt}
{\LARGE\itshape}
[{
    \setlength{\titlewidth}{1.5\titlewidth}
    \titleline*[c]{\titlerule*{\mdseries-}}
}]
\titlespacing*{\part}{0pt}{20pt}{20pt}[0pt]

% Problem and subproblem
\setcounter{secnumdepth}{0}
\newcounter{problem}
\newcounter{subproblem}[problem]
\newbool{appendix}
\gdef\problemName{Problem}
\newcommand{\problem}[2][]{
    \setcounter{equation}{0}
    \ifstrempty{#1}{
        \stepcounter{problem}
    }{
        \setcounter{problem}{#1}
    }
    \ifstrempty{#2}{
        \ifbool{appendix}{
            \section{\hspace{-5mm}\problemName{} \Alph{problem}.}
        }{
            \section{\hspace{-5mm}\problemName{} \arabic{problem}}
        }
    }{
        \ifbool{appendix}{
            \section{\hspace{-5mm}\problemName{} \Alpa{problem}: #2}
        }{
            \section{\hspace{-5mm}\problemName{} \arabic{problem}: #2}
        }
    }
}
\newcommand{\subproblem}[2][]{
    \setcounter{equation}{0}
    \ifstrempty{#1}{
        \stepcounter{subproblem}
    }{
        \setcounter{subproblem}{#1}
    }
    \ifstrempty{#2}{
        \ifbool{appendix}{
            \section{\hspace{-2mm}\Alph{problem}.\arabic{subproblem}}
        }{
            \section{\hspace{-2mm}\arabic{problem}.(\alph{subproblem})}
        }
    }{
        \ifbool{appendix}{
            \section{\hspace{-2mm}\Alph{problem}.\arabic{subproblem}. #2}
        }{
            \section{\hspace{-2mm}\arabic{problem}.(\alph{subproblem}) #2}
        }
    }
}

% Appendix
\newcommand{\myAppendix}{
    \setcounter{section}{0}
    \renewcommand{\sectionname}{Appendix}
    \renewcommand{\thesection}{\Alph{section}}
    \renewcommand{\thesubsection}{\Alph{section}.\arabic{subsection}}
    \renewcommand{\perhapscolon}[1]{\perhapsdot{##1}}
}

% Bookmarks
\hypersetup{
    bookmarksnumbered=true
}

% Reference nome for task and subtask
\crefname{section}{\sectionname}{\sectionnamepl}

\makeatother

%%%%%%%%%%%%%%%%%%%%%%%%%%%%%%%%%%%%%%%%%%%%%%%%%%%

% \everymath{\displaystyle}

% commands for spacing
\newcommand{\Tstrut}{\rule{0pt}{4mm}}         % = `top' strut
\newcommand{\Bstrut}{\rule[-2mm]{0mm}{0mm}}   % = `bottom' strut

% Paired Delimiters {}, (), []
\providecommand\given{} % so it exists
\newcommand\SetSymbol[1][]{
   \nonscript\,#1\vert \allowbreak \nonscript\,\mathopen{}}
\DeclarePairedDelimiterX\Set[1]{\lbrace}{\rbrace}%
 { \renewcommand\given{\SetSymbol[\delimsize]} #1 }
\DeclarePairedDelimiterX{\Bkt}[1]{[}{]}{#1}
\DeclarePairedDelimiterX{\Paren}[1]{(}{)}{#1}
\DeclarePairedDelimiterX{\Abs}[1]{\lvert}{\rvert}{#1}
\newcommand{\PR}[1]{\Pr\Bkt*{#1}}
\newcommand{\overbar}[1]{\mkern 1.5mu\overline{\mkern-1.5mu#1\mkern-1.5mu}\mkern 1.5mu}

\newcolumntype{D}{>{$(}l<{)$}}
\newcolumntype{R}{>{\displaystyle}r}
\newcolumntype{C}{>{\displaystyle}c}
\newcolumntype{L}{>{\displaystyle}l}

%%%%%%%%%%%%%%%%%%%%%%%%%%%%%%%%%%%%%%%%%%%%%%%%%%%

\usepackage{amsfonts}
\usepackage{amsthm}

%%%%%%%%%%%%%%%%%%%%%%%%%%%%%%%%%%%%%%%%%%%%%%%%%%%

% Declare operator and useful math command
\DeclareMathOperator{\EOp}{\mathbb{E}}
\newcommand{\E}[1]{\ensuremath{\EOp\Bkt*{#1}}}
\newcommand{\R}{\mathbb{R}}
\newcommand{\N}{\mathbb{N}}
\newcommand{\Q}{\mathbb{Q}}
\newcommand{\ve}{\varepsilon}
\newcommand{\ie}{\text{ i.e., }}
\newcommand{\st}{\text{ s.t.  }}
\newcommand{\p}[1]{\mathcal{P}(#1)}
% Declare delimiter
\DeclareMathDelimiter{\lVert}
  {\mathopen}{symbols}{"6B}{largesymbols}{"0D}
\DeclareMathDelimiter{\rVert}
  {\mathclose}{symbols}{"6B}{largesymbols}{"0D}
  \DeclarePairedDelimiter\norm{\lVert}{\rVert}

%%%%%%%%%%%%%%%%%%%%%%%%%%%%%%%%%%%%%%%%%%%%%%%%%%%

\theoremstyle{plain}
\newtheorem{corollary}{Corollary}
\newtheorem{proposition}{Proposition}

%%%%%%%%%%%%%%%%%%%%%%%%%%%%%%%%%%%%%%%%%%%%%%%%%%%

\setCJKmainfont{思源宋體 TW}
%\setmainfont{思源宋體 TW}

%%%%%%%%%%%%%%%%%%%%%%%%%%%%%%%%%%%%%%%%%%%%%%%%%%%

\begin{document}
%\begin{CJK*}{UTF8}{bsmi}
\course{Introduction to Mathematical Analysis (I)}
\semester{Fall 2021}
%\teacher{}
%\department{Math}
\assignment{2}
\date{\today}
\student[$\dagger$]{Yu-Chieh Kuo}{B07611039}
\affil[$\dagger$]{Department of Information Management, National Taiwan University}
\ujtitle

% p1
\problem{}
\subproblem{}
Consider the set of all alebraic number $A$ collect all roots of polynomials
with rational coefficients, which mean $A$ is the union of all roots of 
polynomials. For every set with polynomials, there are finite numbers of 
roots, therefore, the union of all polynimials and the union of all roots
are countable as well. That is, the union of all countable sets, as well as 
the set of all algebraic number, $A$, is countable.

\subproblem{}
Let $x = \sqrt{2} + \sqrt{3} \iff x - \sqrt{2} - \sqrt{3} = 0$. Therefore,
we could derive that $x = \sqrt{2} + \sqrt{3}$ is a root of $x^4 - 10x^2 + 1 = 0$.
Hence, $\sqrt{2} + \sqrt{3}$ is algebraic by definition.

% p2
\problem{}
Let $A = \{a_1, a_2, \dots\}$ be an infinite set,
$B = A - \{a_1\} = \{a_2, a_3, \dots\} \subset A$.
Define an one-to-one mapping function
\[
\begin{aligned}[t]
    f:A & \longrightarrow B \\
    a_i & \longmapsto a_{i+1}.
\end{aligned}
\]

By Equivalent Theorem, there exists a one-to-one
function mapping from $B$ onto $A$, we have $A \sim B \ie 
\text{card } A = \text{card } B$.

% p3
\problem{}
Assuming $S$ and $\R$ have the same cardinality, let $\Phi$ be a one-to-one function
from $\R$ onto $S$. Let $a \in \R$, $g_a(\cdot) = \Phi(a)$ be the real-valued
functions in $S$ which corresponds to $a$ in $\R$. Now define $h \in S$ by
$h(x) = 1 + g_x(x), \ x \in \R$. Since $S$ and $\R$ have the same cardinality,
we could find $b \in \R$ such that
\[
    h(b) = \Phi(b) = g_b(b) = 1 + g_b(b),
\]
which leads to the contradiction. Hence, $S$ and $\R$ have the different
cardinality i.e., $S$ and $\R$ are not equivalent.

% p4
\problem{}
Let $S_k = \{ x \in [0,1] : \lvert f(x)\rvert \geq \frac{1}{k}, \ k = 1,2,3,\dots \}$,
we claim that $S_k$ is finite. Asuume it's not finite, we can choose
$n = kM + 1$ such that $\lvert f(x_1)\rvert + \cdots + \lvert f(x_n)\rvert \geq M$,
which leads to a contradiction.
Next, we would like to show that $U = \bigcup\limits_{k=1}^{\infty}S_k$.
\begin{itemize}
    \item ($\bigcup\limits_{k=1}^{\infty}S_k \subseteq U$):
        Suppose $x \in \bigcup\limits_{k=1}^{\infty}S_k$, then there exists
        a $k_x$ such that $x \in S_{k_x} \ie \lvert f(x)\rvert \geq \frac{1}{k_x}$.
        It indicates that $f(x) \neq 0 \implies \bigcup\limits_{k=1}^{\infty}S_k
        \subseteq U$.
    \item ($U \subseteq \bigcup\limits_{k=1}^{\infty}S_k)$:
        Suppose $x \in U$, then there exists $q \neq 0$ such that
        $\lvert f(x) \rvert = q \neq 0$.
        By the archimedean property, we know that there exists $k'
        \geq \frac{1}{q} \ie q \geq \frac{1}{k'}$.
        Therefore, we could derive that
        $x \in S_{k'} \in \bigcup\limits_{k=1}^{\infty}S_k \implies
        U \in \bigcup\limits_{k=1}^{\infty}S_k$.
\end{itemize}
Due to the proofs above, we claim that $U = \bigcup\limits_{k=1}^{\infty}S_k$.
Finally, by the \textbf{Theorem 2.20} in Rudin, since $\N$ is at most countable,
and for every $k \in \N$, $S_k$ is at most countable, 
$U = \bigcup\limits_{k=1}^{\infty}S_k$ is at most countable as well.

% p5
\problem{}
\subproblem{}
Let $A = \{0, 1, \frac{1}{2}, \frac{1}{3},\dots \}$ and $B =\{ \frac{1}{2}, \frac{1}{3},
\dots \}$. Define
\[
    f(x) = \begin{cases}
        \frac{1}{2} & \text{if } x = 0 \\
        \frac{1}{n+2} & \text{if } x = \frac{1}{n}, \ n = 1,2,3,\dots.
    \end{cases}
\]
Clearly, $f$ is a one-to-one mapping from the set $A$ onto set $B$.

To extend this mapping to a function $f : [0,1] \to (0,1)$ that is one-to-one
and onto, we re-define $f$ as
\[
    f(x) = \begin{cases}
        \frac{1}{2} & \text{if } x = 0 \\
        \frac{1}{n+2} & \text{if } x = \frac{1}{n}, \ n = 1,2,3,\dots \\
        x & \text{if } x \in [0,1] \setminus A.
    \end{cases}
\]
Based on the definition above, we could find that $f$ is a one-to-one
mapping from $[0,1]$ onto $(0,1)$. Thus, $[0,1]$ and $(0,1)$
have the same cardinality.

\subproblem{}
Divide $[0, \infty)$ into two intervals and follow the concept of 5.(a).
Let $A = \{0, 1, 2, \dots\} = 0 \cup \mathbb{Z}_+$ and $B = [0,\infty) \setminus A$, we define
\[
    f(x) = \begin{cases}
        x+1 & \text{if } x \in A \\
        x & \text{if } x \in (0,\infty) \setminus \mathbb{Z}_+,
    \end{cases}
\]
where $f$ is a one-to-one mapping from $[0, \infty)$ onto $(0, \infty)$.
Thus, we conclude that $[0, \infty)$ and $(0, \infty)$ have the same cardinality.

To conclude that all intervals in the extended real number system have the
same cardinality, we need to prove that $(a,b) \sim [a,b]$ for all $a, b \in \R$.
From the previous proof, we need to prove that $(-\infty,0]$ and $(-\infty,0)$
have the same cardinality to extend the system to all real number. Based on
the same idea, define a function $g: (-\infty,0] \to (-\infty,0)$ as
\[
    g(x) = \begin{cases}
        x-1 & \text{if } x \in 0 \cup \mathbb{Z}_- \\
        x & \text{if } x \in (-\infty, 0) \setminus \mathbb{Z}_-,
    \end{cases}
\]
and we can clearly conclude that $g$ is a one-to-one mapping from $(-\infty, 0]$
onto $(-\infty, 0)$. As a result, we extend the system for all real numbers
i.e., for every open, half-open, and closed intervals with real number boundaries,
they have the same cardinality.

\subproblem{}
Since the proofs between $(0,1] \sim (0,1] \times (0,1]$ and $(0,1) \sim (0,1) \times (0,1)$ are equivalent,
to prove that $(0,1] \sim (0,1] \times (0,1]$, we could construct two
one-to-one functions $f:(0,1) \to (0,1) \times (0,1)$
and $g: (0,1) \times (0,1) \to (0,1)$. If we could find such $f$ and $g$,
then we can say that $(0,1) \subset \R$ and $(0,1) \times (0,1) \subset \R^2$
have the same cardinality i.e.,
$(0,1] \subset \R$ and $(0,1] \times (0,1] \subset \R^2$
have the same cardinality as well.
Define
\[
\begin{aligned}[t]
    f:(0,1) & \longrightarrow (0,1) \times (0,1) \\
    x & \longmapsto (x, \frac{1}{2}).
\end{aligned}
\]
Clearly $f$ is a one-to-one injection from $(0,1)$ to $(0,1) \times (0,1)$.

Next, define
\[
\begin{aligned}[t]
    g: (0,1] \times (0,1) & \longrightarrow (0,1) \\
    (x,y) = (0.x_1x_2\dots, 0.y_1y_2\dots) & \longmapsto 0.x_1y_1x_2y_2\dots.
\end{aligned}
\]
Note that the decimal representation is unique i.e., we don't say that 
$0.1999\dots = 0.2$. Hence, $g$ is one-to-one mapping from $(0,1)$ to $(0,1) \times (0,1)$.
With the injection from both sides, we could conclude that $(0,1) \sim (0,1) \times (0,1)$
i.e., $(0,1] \subset \R$ and $(0,1] \times (0,1] \subset \R^2$ have the
same cardinality.



% p6
\problem{}
\subproblem{}
Let $f: S \to \mathcal{P}(S)$ and consider the set $A:\{ x \in S \ | \ x \notin f(x) \}$.
Assume that $f$ is surjective, then there exists $y \in S$ such that $f(y) = A$.
However, $y \in B \iff y \notin f(y) = B$, which leads to a contradictory and implies
that $f$ is not surjective. On the other hands, if we define
$g : S \to \mathcal{P}(S)$ is an injective mapping with $x \mapsto {x}$, then
$g$ is a possible mapping from $S$ to $\mathcal{P}(S)$. As a result,
we must have the cardinality of $S$ is less than the cardinality of $\mathcal{P}(S)$.

\subproblem{}
Let $S$ be the set of all sets, and $\mathcal{P}(S)$ be the set of its all subsets.
Since $S$ is "the set of all sets", it must satisfy that $\lvert S \rvert
\geq \mathcal{P}(S)$, which contradicts to the Cantor's Theorem. Consequently,
"the set of all sets" makes no sense at all.

% p7
\problem{}
\subproblem{}
Given by the definition of $g$, consider its inverse function $g^{-1}: A' \to B'$
such that $g^{-1}(g(x)) = x \ \forall x \in B'$. Next, we would like to check
whether $g^{-1}$ is one-to-one and onto.
\begin{enumerate}
    \item One-to-one:
        For arbitraty $g(x), g(y) \in A'$, we have
        \[
            \begin{array}{RL}
                & g^{-1}(g(x)) = g^{-1}(g(y)) \\
                \implies & x = y \\
                \implies & g(x) = g(y) \\
                \implies & g^{-1} \text{ is one-to-one.}
            \end{array}
        \]
    \item Onto:
        For any $b \in B'$, since $g$ is onto, there exists $a \in A'$
        such that $g(b) = a$. Moreover,
        \[
            \begin{array}{RL}
                & g(b) = a \\
                \implies & g^{-1}(g(b)) = g^{-1}(a) \\
                \implies & b = g^{-1}(a) \\
                \implies & g^{-1} \text{ is onto.}
            \end{array}
        \]
\end{enumerate}
Define $h: X \to Y$ such that
\[
    h(x) = \begin{cases}
        f(x) & \text{if } x \in A \\
        g^{-1}(x) & \text{if } x \in A'.
    \end{cases}
\]
Since $f$ and $g^{-1}$ are one-to-one and onto mapping, it leads to
$h: X \to Y$ is also one-to-one and onte, which implies $X \sim Y$.

\subproblem{}
For $A_1 = \emptyset$, it shows that $g(Y) = X$. Consequently, we could
take $g^{-1}$ as a one-to-one and onto function and conclude $X \sim Y$.

For $A_1 \neq \emptyset$, by induction, consider the case of $n=1$ and $n = 2$,
\[
    \begin{array}{RL}
        & x \in A_1 \\
        \implies & x \notin g(Y) \\
        \implies & x \notin g(f(A_1)) \quad \because f(A_1) \subseteq Y \\
        \implies & x \notin A_2.
    \end{array}
\]
Here it shows that $A_1, A_2$ are disjoint pairwise.

By induction hypothesis, suppose that in the case of $n = k$, $A_k$ is disjoint
from $A_1, A_2,\dots,A_{k-1}$. Here we define $x' = g(f(x))$ for the future
use, and it leads to
\[
    \begin{array}{RL}
        & x \in A_k \\
        \iff & g(f(x)) \in g(f(A_k)) \\
        \iff & x' \in g(f(A_k)) \\
        \iff & x' \in A_{k+1}.
    \end{array}
\]
And for $x' \in A_{k+1}$, we have that
\[
    \begin{array}{RL}
        & x' \in A_{k+1} \\
        \iff & x \in A_k \\
        \iff & x \notin A_1, x \notin A_2, \dots, x \notin A_{k-1} \\
        \iff & x' \notin A_2, x' \notin A_3, \dots, x' \notin A_k.
    \end{array}
\]
For the case of $n = k+1$, we have
\[
    \begin{array}{RL}
        & x' \in A_{k+1} \\
        \implies & x' \in g(f(A_k)) \\
        \implies & x' \in g(Y) \quad \because f(A_k) \subseteq Y \\
        \implies & x' \notin A_1.
        
    \end{array}
\]
Consequently, for the case of $x' \in A_{k+1}$, it implies that 
$x'$ doesn't belong to $A_1, A_2, \dots, A_k$. Hence, we prove that 
$\{ A_n : n \in \N \}$ is a pairwise disjoint collection of subsets of $X$,
and $\{ f(A_n) : n \in \N \}$ is a similar collection in $Y$ since $f$
is a one-to-one function.

\subproblem{}
Take any $b \in B$, there exists $n \in \N$ such that $b \in f(A_n) \subset B$,
where $n \in \N$.
By the definition of $A_n$ and $f(A_n)$ for $n \in \N$, there exists $a \in A_n \subset A$
,$n \in \N$, such that $f(a) = b$, which indicates that $f$ maps $A$ onto $B$.

\subproblem{}
Take any $a \in A'$, we have 
\[
    \begin{array}{RL}
        & a \in A' \\
        \implies & a \notin A \\
        \implies & a \notin A_1 \quad \because A_1 \subset A \\
        \implies & a \in g(Y) \\
        \implies & \exists \ b \in Y \st g(b) = a.
    \end{array}
\]
Suppose that $b \notin B' \ie b \in B$, we have $b \in f(A_n)$ for some 
$n \in \N$. Since $f$ is a onto mapping, there exists $a' \in A_n$
such that $f(a') = b$. But here 
$a = g(b) = g(f(a')) \in g(f(A_n)) = A_{n+1} \subset A$ for $n \in \N$,
which is a contradiction. Consequently $b$ should be in $B'$. Therefore
we prove that $g$ maps $B'$ onto $A'$ i.e., for any $a \in A'$, there
exists $b \in B'$ such that $g(b) = a$.

\subproblem{}
Define $A_1 = X \setminus g(Y) = X \setminus [0,\frac{1}{2}] = (\frac{1}{2},1]$.
By induction, define $A_2, A_3, \dots, A_n(n \in \N)$ as
\[
    \begin{array}{RRCLL}
        A_2 & = & g(f(A_1)) & = & (\frac{1}{8}, \frac{1}{4}] \\ [4mm]
        A_3 & = & g(f(A_2)) & = & \left(\frac{1}{32},  \frac{1}{16}\right] \\
        & &                   \vdots & & \\
        A_n & = & g(f(A_{n-1})) & = & \left(\frac{1}{2\cdot r^{n-1}}, \frac{1}{4^{n-1}}
        \right],
    \end{array}
\]
and $A = \bigcup_{n=1}^{\infty}A_n = \bigcup_{n=1}^{\infty}\left(\frac{1}{2\cdot4^{n-1}},
\frac{1}{4^{n-1}}\right]$. Therefore, we could define $h:[0,1] \to [0,1]$ as 
\[
    h(x) = \begin{cases}
        \frac{x}{2} & \text{if } x \in \bigcup_{n=1}^\infty\left(\frac{1}{2\cdot4^{n-1}},
        \frac{1}{4^{n-1}}\right] \\
        2x & \text{otherwise.}
    \end{cases}
\]

% p8
\problem{}
\subproblem{}
For all $t \in T$, we have the preimage $f^{-1}(t) \neq \emptyset \subset S$.
By Axiom of Choice, there exists a function $U$ such that $U(t) \in f^{-1}(t)$,
where $U$ is one-to-one.

\subproblem{}
By the fact deriving from 8.(a), we know that there exists functions
$U: S \to T$ and $V: T \to S$, where $U$ and $V$ are one-to-one.
By equivalent theorem, we conclude that $S \sim T$ i.e., $S$ and $T$
have the same cardinality.

%\end{CJK*}
\end{document}

