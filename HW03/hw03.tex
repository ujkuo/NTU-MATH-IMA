\documentclass[a4paper]{article}
%The main tex begins from line 188
\usepackage[top = 2.5cm, bottom = 2.5cm, left = 2cm, right = 2cm]{geometry} 
\usepackage{amsmath} % For the usage of \because and \therefore
\usepackage{amssymb}
\usepackage{fancyhdr}
\usepackage{lastpage}
\usepackage{etoolbox}
\usepackage{indentfirst}
\usepackage{tabularx}
\usepackage{mathtools}
\usepackage{booktabs}
\usepackage{authblk}
\usepackage[calcwidth]{titlesec}
\usepackage{bookmark}
\usepackage[capitalize]{cleveref}
\usepackage{array}
\usepackage[english]{babel}
\usepackage{enumitem} % For the usage of labeling in enumerate and itemize
%\usepackage[utf8]{inputenc}
%\usepackage{CJKutf8}
\usepackage{xeCJK}

%%%%%%%%%%%%%%%%%%%%%%%%%%%%%%%%%%%%%%%%%%%%%%%%%%%

\makeatletter
% Macros for setting basic info
\gdef\@uni{National Taiwan University}
\gdef\@department{Information Management}
\gdef\@me{Yu-Chieh Kuo}

\newcommand{\course}[2][]{
    \ifstrempty{#1}{
        \gdef\shortcourse{#2}
    }{
        \gdef\shortcourse{#1}
    }
    \gdef\@course{#2}
}
\newcommand{\teacher}[1]{\gdef\@teacher{#1}}
\newcommand{\semester}[1]{\gdef\@semester{#1}}
\newcommand{\uni}[1]{\gdef\@uni{#1}}
\newcommand{\department}[1]{\gdef\@department{#1}}
\newcommand{\student}[3][]{
    \ifstrempty{#1}{
        \author{\textbf{#2} \textbf{#3}}
    }{
        \author[#1]{\textbf{#2} \textbf{#3}}
    }
}
\newcommand{\assignment}[2][Homework]{
    \ifstrempty{#2}{
        \gdef\@assignment{#1}
    }{
        \gdef\@assignment{#1 #2}
    }
}

% page style for title page
\fancypagestyle{title}{
    \fancyhf{}
    \renewcommand{\headrulewidth}{0pt}
    \cfoot{\footnotesize Page {\thepage} of \pageref{LastPage}} 
}
\newcommand{\ujtitle}{
    \thispagestyle{title}
    \noindent\begin{tabularx}{\linewidth}{Xr} % This is a simple tabular environment to align your text nicely 
    {\large \bf{\@course}} \\
    National Taiwan University, \@semester  \\
    \bottomrule % \hline produces horizontal lines.
    \end{tabularx}
    
    \vspace*{0.3cm}

    \begin{center}
        {\huge\bf\textsc{\@assignment}}\\[3mm]
        {\@author}
    \end{center}

    \vspace*{0.3cm}
}
\newcommand{\mymaketitle}{
    \thispagestyle{title}
    \vspace*{-15mm}
    \noindent\begin{tabularx}{\linewidth}{Xr}
                                & \textsl{National Taiwan University} \\
        \large\textbf{\@semester,} & \textsl{Department of \@department}         \\
        \large\textbf{\@course} & \textsl{Prof. \@teacher}\Bstrut\\
        \bottomrule
    \end{tabularx}

    \vspace*{10mm}

    \begin{center}
        {\huge\textsc{\@assignment{}}}\\[6mm]
        {\@author}
    \end{center}

    \vspace*{1cm}
}

% page style for contents
\pagestyle{fancy}
\fancyhf{}
\setlength{\headheight}{13.6pt}
\lhead{\small {\@course}: {\@assignment}}
\rhead{\small {\@me}}
\cfoot{\small {Page {\thepage} of \pageref{LastPage}}} 

% Style for part, section, and subsection
% Part
\titleclass{\part}{straight}
\titleformat{\part}
[display]
{\centering\bfseries}
{\setlength{\titlewidth}{1.5\titlewidth}\large\titleline*[c]{\titlerule[.6pt]}\vspace{3pt}\MakeUppercase{\partname} \thepart}
{0pt}
{\LARGE\itshape}
[{
    \setlength{\titlewidth}{1.5\titlewidth}
    \titleline*[c]{\titlerule*{\mdseries-}}
}]
\titlespacing*{\part}{0pt}{20pt}{20pt}[0pt]

% Problem and subproblem
\setcounter{secnumdepth}{0}
\newcounter{problem}
\newcounter{subproblem}[problem]
\newbool{appendix}
\gdef\problemName{Problem}
\newcommand{\problem}[2][]{
    \setcounter{equation}{0}
    \ifstrempty{#1}{
        \stepcounter{problem}
    }{
        \setcounter{problem}{#1}
    }
    \ifstrempty{#2}{
        \ifbool{appendix}{
            \section{\hspace{-5mm}\problemName{} \Alph{problem}.}
        }{
            \section{\hspace{-5mm}\problemName{} \arabic{problem}}
        }
    }{
        \ifbool{appendix}{
            \section{\hspace{-5mm}\problemName{} \Alpa{problem}: #2}
        }{
            \section{\hspace{-5mm}\problemName{} \arabic{problem}: #2}
        }
    }
}
\newcommand{\subproblem}[2][]{
    \setcounter{equation}{0}
    \ifstrempty{#1}{
        \stepcounter{subproblem}
    }{
        \setcounter{subproblem}{#1}
    }
    \ifstrempty{#2}{
        \ifbool{appendix}{
            \section{\hspace{-2mm}\Alph{problem}.\arabic{subproblem}}
        }{
            \section{\hspace{-2mm}\arabic{problem}.(\alph{subproblem})}
        }
    }{
        \ifbool{appendix}{
            \section{\hspace{-2mm}\Alph{problem}.\arabic{subproblem}. #2}
        }{
            \section{\hspace{-2mm}\arabic{problem}.(\alph{subproblem}) #2}
        }
    }
}

% Appendix
\newcommand{\myAppendix}{
    \setcounter{section}{0}
    \renewcommand{\sectionname}{Appendix}
    \renewcommand{\thesection}{\Alph{section}}
    \renewcommand{\thesubsection}{\Alph{section}.\arabic{subsection}}
    \renewcommand{\perhapscolon}[1]{\perhapsdot{##1}}
}

% Bookmarks
\hypersetup{
    bookmarksnumbered=true
}

% Reference nome for task and subtask
\crefname{section}{\sectionname}{\sectionnamepl}

\makeatother

%%%%%%%%%%%%%%%%%%%%%%%%%%%%%%%%%%%%%%%%%%%%%%%%%%%

% \everymath{\displaystyle}

% commands for spacing
\newcommand{\Tstrut}{\rule{0pt}{4mm}}         % = `top' strut
\newcommand{\Bstrut}{\rule[-2mm]{0mm}{0mm}}   % = `bottom' strut

% Paired Delimiters {}, (), []
\providecommand\given{} % so it exists
\newcommand\SetSymbol[1][]{
   \nonscript\,#1\vert \allowbreak \nonscript\,\mathopen{}}
\DeclarePairedDelimiterX\Set[1]{\lbrace}{\rbrace}%
 { \renewcommand\given{\SetSymbol[\delimsize]} #1 }
\DeclarePairedDelimiterX{\Bkt}[1]{[}{]}{#1}
\DeclarePairedDelimiterX{\Paren}[1]{(}{)}{#1}
\DeclarePairedDelimiterX{\Abs}[1]{\lvert}{\rvert}{#1}
\newcommand{\PR}[1]{\Pr\Bkt*{#1}}
\newcommand{\overbar}[1]{\mkern 1.5mu\overline{\mkern-1.5mu#1\mkern-1.5mu}\mkern 1.5mu}

\newcolumntype{D}{>{$(}l<{)$}}
\newcolumntype{R}{>{\displaystyle}r}
\newcolumntype{C}{>{\displaystyle}c}
\newcolumntype{L}{>{\displaystyle}l}

%%%%%%%%%%%%%%%%%%%%%%%%%%%%%%%%%%%%%%%%%%%%%%%%%%%

\usepackage{amsfonts}
\usepackage{amsthm}

%%%%%%%%%%%%%%%%%%%%%%%%%%%%%%%%%%%%%%%%%%%%%%%%%%%

% Declare operator and useful math command
\DeclareMathOperator{\EOp}{\mathbb{E}}
\newcommand{\E}[1]{\ensuremath{\EOp\Bkt*{#1}}}
\newcommand{\R}{\mathbb{R}}
\newcommand{\N}{\mathbb{N}}
\newcommand{\Q}{\mathbb{Q}}
\newcommand{\Z}{\mathbb{Z}}
\newcommand{\PS}[1]{\mathcal{P}(#1)}
\newcommand{\ve}{\varepsilon}

\newcommand{\sst}{\subset}
\newcommand{\sse}{\subseteq}

\newcommand{\ie}{\text{ i.e., }}
\newcommand{\st}{\text{ s.t.  }}


% Declare delimiter
\DeclareMathDelimiter{\lVert}
  {\mathopen}{symbols}{"6B}{largesymbols}{"0D}
\DeclareMathDelimiter{\rVert}
  {\mathclose}{symbols}{"6B}{largesymbols}{"0D}
\DeclarePairedDelimiter\norm{\lVert}{\rVert}

%%%%%%%%%%%%%%%%%%%%%%%%%%%%%%%%%%%%%%%%%%%%%%%%%%%

\theoremstyle{plain}
\newtheorem{corollary}{Corollary}
\newtheorem{proposition}{Proposition}

%%%%%%%%%%%%%%%%%%%%%%%%%%%%%%%%%%%%%%%%%%%%%%%%%%%

\setCJKmainfont{思源宋體 TW}
%\setmainfont{思源宋體 TW}

%%%%%%%%%%%%%%%%%%%%%%%%%%%%%%%%%%%%%%%%%%%%%%%%%%%

\begin{document}
%\begin{CJK*}{UTF8}{bsmi}
\course{Introduction to Analysis I}
\semester{Fall 2021}
%\teacher{}
\department{}
\assignment{3}
\date{\today}
\student[$\dagger$]{Yu-Chieh Kuo}{B07611039}
\affil[$\dagger$]{Department of Information Management, National Taiwan University}
\ujtitle

% p1
\problem{}

% p1-1
\subproblem{}
To prove $d_2$ is a metric on $X$, we need to show that
\begin{enumerate}
    \item $d_2(p,q) > 0$ if $p \neq q$ and $d(p,p) = 0$.
    \item $d_2(p,q) = d_2(q,p)$.
    \item $d_2(p,q) \leq d_2(p,r) + d_2(r,q) \ \forall r \in X$.
\end{enumerate}

Also, since $d$ is a metric on $X$, we have $d(p,q) > 0$ 
if $p \neq q$ and $d(p,p) = 0$.

To 1., we have
\[
    \begin{array}{RCCL}
        d_2(p,q) & = & \frac{d(p,q)}{1+d(p,q)} & > 0 \\
        d_2(p,p) & = & \frac{d(p,p)}{1+d(p,p)} & = 0.
    \end{array}
\]

To 2., we have
\[
    d_2(p,q) = \frac{d(p,q)}{1+d(p,q)} = \frac{d(p,q)}{1+d(p,q)}
    = d_2(q,p).
\]

To 3., if $d(p,q) > 0$, then
\[
    \begin{array}{RCCL}
        d_2(p,q) & = & \frac{d(p,q)}{1+d(p,q)} & = 1 - \frac{1}{1+d(p,q)} \\
        & \leq & 1 - \frac{1}{1+d(p,r)+d(r,q)} & =  \frac{d(p,r)+d(r,q)}{1+d(p,r)+d(r,q)} \\
        & \leq & \frac{d(p,r)}{1+d(p,r)} + \frac{d(r,q)}{1+d(r,q)} & = d_2(p,r) + d_2(r,q).
    \end{array}
\]
This property holds when $d(p,q) = 0 $ as well. 

The significance of $d_2$ indicates that we can define another metric
based on a defined metric.

% p1-2
\subproblem{}
As 1.(a), we could find $\rho$ is a metric by checking those three properties:
\begin{enumerate}
    \item $\rho(a,b) > 0$ if $a \neq b$ and $\rho(a,a) = 0$.
    \item $\rho(a,b) = \rho(b,q)$.
    \item $\rho(a,b) \leq \rho(a,c) + \rho(c,b) \ \forall c \in \{0, 1\}^{\N}$.
\end{enumerate}
To 1., we define $\N_+$ as the set containing positive natural numbers and
positive infinity.
If $a = b$, then $a_i = b_i \implies \rho(a,b) = 0$, where $i \in \N_+$.
If $a \neq b$, then there exists some $i \in \N_+$ is a positive number such that
$a_i \neq b_i$, which implies $\lvert a_i - b_i \rvert > 0$. Therefore,
$\rho(a,b)>0$.

To 2., since $|a_i - b_i| = |b_i - a_i|$, we have
\[
    \rho(a,b) = \sum^{\infty}_{i=1}\frac{|a_i - b_i|}{2^i}
    = \sum^{\infty}_{i=1} \frac{|b_i - a_i|}{2^i}
    = \rho(b,a).
\]

To 3., since $a_i, b_i \in \R \ \forall i \in \N_+$, and the triangle
inequality holds for real numbers, we can say that $\forall c_i \in
\{0,1\}^{\N}$, $|a_i + b_i| \leq |a_i - c_i| + |c_i - b_i|$ holds.
Thus,
\[
    \begin{array}{RCL}
        \rho(a,b) & = & \sum^{\infty}_{i=1}\frac{|a_i-b_i|}{2^i} \\
        & \leq & \sum^{\infty}_{i=1}\frac{|a_i-c_i| + |c_i - b_i|}{2^i} \\
        & = & \sum^{\infty}_{i=1}\frac{|a_i-c_i|}{2^i} 
        + \sum^{\infty}_{i=1}\frac{|c_i-b_i|}{2^i} \\
        & = & \rho(a,c) + \rho(c,b).  
    \end{array}
\]
By checking the definition of metric, we can find that $\rho$ is a metric.

After substituting $\{0,1\}^{\N}$ into the infinite product space
$Y = X_1 \times X_2 \times \cdots$, and substituting $|a_i - b_i|$
into $d_i(a_i, b_i)$, to check new $\rho$ is a metric or not, we could
still follow the same process as above.

At first, if $a = b$, then $a_i = b_i \forall i \in \N_+$, which
implies $d_i(a_i, b_i) = 0 \forall i \in \N_+$. Hence, it's clearly that
$\rho(a,b) = \sum^{\infty}_{i=1} \frac{d_i(a_i,b_i)}{2^i} = 0$.
If $a \neq b$, then there exists some $i \in \N_+$ is a positive number such that
$a_i \neq b_i$, which implies $d_i(a_i,b_i)>0$ for this $i$. Therefore,
it's clearly that $\rho(a,b)>0$.

Secondly, since $d_i$ is a metric for $i \in \N_+$, we have
\[
    \rho(a,b) = \sum^{\infty}_{i=1} \frac{d_i(a_i,b_i)}{2^i}
    = \sum^{\infty}_{i=1} \frac{d_i(b_i,a_i)}{2^i} = \rho(b,a).
\]

Lastly, since $d_i$ is a metric for $i \in \N_+$, the triangle inequality holds.
Therefore,
\[
    \begin{array}{RCL}
        \rho(a,b) & = & \sum^{\infty}_{i=1}\frac{d_i(a_i-b_i)}{2^i} \\
        & \leq & \sum^{\infty}_{i=1}\frac{d_i(a_i-c_i)+ d_i(c_i-b_i)}{2^i} \\
        & = & \sum^{\infty}_{i=1}\frac{d_i(a_i-c_i)}{2^i} 
        + \sum^{\infty}_{i=1}\frac{d_i(c_i-b_i)}{2^i} \\
        & = & \rho(a,c) + \rho(c,b).
    \end{array}
\]

By checking the definition of metric, we can find that $\rho$ is a metric.

% p2
\problem{}
First, we show that $E'$ is closed.

By {\bf Definition 2.18} in Rudin, it suffices to say every limit point
of $E'$ is a limit point of $E$.
Given a limit point $p \in E'$, every neighborhood $V$ of $p$ contains
a point $p'$ such that $p' \in E'$. Since $p'$ is a limit point of $E$,
there exists an open neighborhood $U$ of $p'$ contains $q \neq p'$
such that $q \in E$, where
\[
    U = V \cap B\left(p',\frac{1}{2}d(p,p')\right) \sse V.
\]
(Note that $B(a,b)$ means the open ball centering at $a$ with radius $b$.)\\
Therefore, since $q \neq p$, $q \in U \sse V$, and $q \in E$, $p$ is a 
limit point of $E$.

Next, we would like to show that $X - E'$ is open, which is equivalent to 
prove $E'$ is close.
Given a point $p \in X- E'$, there exists an open neighborhood $V$ of $p$
contains no points $q \neq p$ such that $q \in E$.
Next, to show $V$ is an open neighborhood of $p$ such that $V \sse X-E'$,
we assume by contradiction that there exists a limit point $q$ of $E$
such that $q \neq p$ and $q \in V$, then here we could contruct
a neighborhood $U$ as
\[
    U = V \cap B\left(q,\frac{1}{2}d(p,q)\right) \sse V,
\]
where $U$ is an open neighborhood of $q$ containing on points of $E$,
leading to a contrary. Consequently, $V \sse X-E'$ is an open neighborhood
of $p \in X-E'$ i.e., $X-E'$ is open, equivalent to $E'$ is closed.

Second, we want to prove that $E$ and $\bar E$ have the same limit points.
It is equivalent to prove that whether $E' = \bar E'$.
\begin{itemize}
    \item ($E' \sse \bar E'$): 
        Since $E \sse \bar E$, $E' \sse \bar E'$ obviously.
    \item ($E' \subseteq \bar E'$):
        Given a limit point $p$ of $\bar E = E \cup E'$. If $p$
        is a limit point of $E$, the proof is done; if $p$ is a limit point
        of $E'$, since $E'$ is closed, it has to be that $p \in E'$
        or $p$ is a limit point of $E$.
        In any case, $E' subseteq \bar E'$ is true.
\end{itemize}

Lastly, the problem comes to whether $E$ and $E'$ always have the same limit point.
It seems to have an easy yes; however, the answer is false. Consider
\[
    E = \{\frac{1}{n}: n \in \Z_+ \} \sse \R.
\]
Here $E' = \{0\}$ but $(E')' = \emptyset$.

% p3
\problem{}
\begin{enumerate}[label = (\alph*)]
    \item This problem is equivalent to show that $E^\circ \sse (E^\circ)^\circ$.
        Given any $x \in E^\circ$, there exists $r>0$ such that $B(x,r) \sse E$. 
        Since $B(x,r)$ is an open set, all points in $B(x,r)$ are interior points.
        That is, $\forall y \in B(x,r), \ \exists s>0 \st B(y,s) \sse B(x,r) \sse E$.
        This implies that every $y \in B(x,r)$ is an interior point of $E$ i.e.,
        $B(x,r) \sse E^\circ \implies x \in (E^\circ)^\circ \iff
        E^\circ \sse (E^\circ)^\circ$.
    \item By {\bf Definition 2.18} in Rudin, since $E$
        is open, every point of $E$ is an interior point of $E$ i.e.
        $E \sse E^\circ$. Also, $E^\circ \sse E$ since all points in
        $E^\circ$ must be in $E$. Thus, $E = E^\circ$.
    \item $G \sse E \implies G^\circ \sse E^\circ$, and $G = G^\circ$ by (b).
        Hence $G = G^\circ \sse E^\circ$.
    \item 
        \begin{itemize}
            \item ($(E^\circ)^c \sse \bar E^c$):
                Given any $x \in (E^\circ)^c$, if $x \notin E$, then 
                $x \in E^c \sse \bar E^c$; if $x \in E$, then every neighborhood $U$
                of $x$ must satisfy $U \cap E^c \neq \emptyset$ i.e., $x$ is a 
                limit point of $E^c$, which means $x \in (E^c)' \sse \bar E^c$.
                For both cases, $x \in \bar E^c$ at all.
            \item ($\bar E^c \sse (E^\circ)^c$): 
                Given any $x \in \bar E^c$, it shows that either $x \in E^c$
                or $x \in (E^c)'$. If $x \in E^c$, then 
                $x \neq E \implies x \neq E^\circ \ie x \in (E^\circ)^c$.
                On the other hand, if $x \in (E^c)'$, then every neighborhood $V$
                of $x$ must satisfy $V \cap E^c \neq \emptyset$. Hence, $x$ is not
                an interior point of $E$, i.e., $E \notin E^\circ \implies 
                x \in (E^\circ)^c$.
        \end{itemize}
    \item The answer is {\bf NO}. Consider $X$ is a metric space in $\R$ and 
        $E = \Q \sse X$. We have $E^\circ = \emptyset$ since $\bar \Q$ is dense in $\R$,
        and $(\bar E)^\circ = R$ since $\Q$ is dense in $\R$ and $\R$ is open.
    \item The answer is {\bf NO}. Following by the setting in (e),
        we have $\bar E = \R$ since $\Q $ is dense in $\R$, and $\bar{E^\circ}
         = \bar \emptyset = \emptyset$ since $\bar \Q$ is dense in $\R$.
\end{enumerate}

% p4
\problem{}
\subproblem{}
Consider $E$ be the set of points having only rational coordinates.
Since $\Q$ is countable, $E = \Q^k$ is also countable.

Now, given any ${\bf p} = (p_1,\dots,p_k) \in \R^k$, we would like
to show that ${\bf p}$ is a limit point of $E$.
Consider any open neighborhood $N_r({\bf p})$ with $r > 0$. As $\Q$
is dense in $\R$, every coordinate of ${\bf p}$ is a limit point of
$\Q$.


%\end{CJK*}
\end{document}

