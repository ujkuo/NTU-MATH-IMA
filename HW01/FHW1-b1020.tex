\documentclass[a4paper, 12pt]{article}%twoside,
\usepackage{amssymb}
\usepackage{amsmath}
\usepackage{xeCJK}
\usepackage{array}
\setCJKmainfont{思源宋體 TW}
\XeTeXlinebreaklocale "zh"
\XeTeXlinebreakskip = 0pt plus 1pt %這兩行一定要加,中文才能自動換行

\newcommand{\be}{\begin{equation}}
\newcommand{\ee}{\end{equation}}
\newcommand{\bear}{\begin{eqnarray*}}
\newcommand{\eear}{\end{eqnarray*}}
\newcommand{\bearr}{\begin{eqnarray}}
\newcommand{\eearr}{\end{eqnarray}}
\newcommand{\bi}{\begin{itemize}}
\newcommand{\ei}{\end{itemize}}
\newcommand{\bn}{\begin{enumerate}}
\newcommand{\en}{\end{enumerate}}
\newcommand{\bd}{\begin{description}}%\bd[]
\newcommand{\ed}{\end{description}}
\newcommand{\I}{\item}
\newcommand{\varep}{\varepsilon}
\newcommand{\mc}[1]{{\mathcal #1}}
\newcommand{\mf}[1]{{\mathfrak #1}}
\newcommand{\mb}[1]{{\mathbf #1}}
\newcommand{\bb}[1]{{\mathbb #1}}

%\pagestyle{empty} %後的指令不會執行(ignored),可供撰寫comment
\setlength{\textwidth}{18.5cm}
\setlength{\oddsidemargin}{-1.7cm}
\setlength{\evensidemargin}{1pt}
\setlength{\topmargin}{-3cm}
\setlength{\textheight}{26.5cm}

\newcolumntype{R}{>{\displaystyle}r}
\newcolumntype{C}{>{\displaystyle}c}
\newcolumntype{L}{>{\displaystyle}l}

\begin{document}
\begin{center}
{INTRODUCTION TO MATHEMATICAL ANALYSIS (I), FALL SEMESTER, 2021} \\  %[.2cm]
{\sc Homework 1, \rm{\today} }
\end{center}
\medskip
{Dept. \underline{\quad IM. \quad}\qquad 
	Name \underline{\quad 郭宇杰\quad}\qquad 
	Student ID\underline{\quad B07611039\quad} \qquad 
	Grade \hrulefill }

\bn
\I {[R]} Ex. 1.6 (c) and (d) (definition and properties of exponential functions)\\
{\bf Ans.} 可使用中文。於截止時限前上傳pdf檔至NTU COOL完成繳交作業,其中ID 改為自己的學號。 \hfill $\square$

\I {[R]} Ex. 1.7 (logarithm of $y>0$ to the base $b>1$). \\
{\bf Ans.}

\I {[A]} Ex. 1.23 \& 1.48, [R] Ex. 15 (real and complex Lagrange identities)
\bn
\I
Prove Lagrange identity for real numbers:
\begin{equation}
		\left( \sum_{k=1}^n a_k b_k\right)^2 =
		\left( \sum_{k=1}^n a_k^2\right) \left( \sum_{k=1}^n b_k^2\right)
		- \sum_{1\le k < j \le n} \left(  a_kb_j-a_jb_k \right)^2. 
		\label{L}
\end{equation}
In fact, the left hand side of (\ref{L}) is the inner product 
$\langle {\,\mathbf a, \mathbf b} \rangle$ of ${\mathbf a}=(a_1, \ldots, a_n)$ and
${\mathbf b}=(b_1, \ldots, b_n)$ in $\mathbb R^n$.

\I
Define the inner product of ${\bf a}=(a_1, \ldots, a_n)$ and
${\bf b}=(b_1, \ldots, b_n)$ in $\mathbb C^n$ by
\[
\langle {\,\mathbf a, \mathbf b} \rangle
=\sum_{k=1}^n a_k \bar{b}_k.
\]
Derive a Lagrange identity of complex numbers by
starting with $| \langle {\,\mathbf a, \mathbf b} \rangle \,|^2$.
\I
Derive the real and complex Cauchy-Schwarz inequalities from
the respective real and complex Lagrange identities.
Under what conditions do equalities hold, respectively?
\en
{\bf Ans.}
\bn
\I
We start deriving from the left side.
\[
    \begin{array}{RL}
        & \left( \sum_{k=1}^n a_k^2\right) \left( \sum_{k=1}^n b_k^2\right) - \sum_{1 \leq k < j \leq n}(a_kb_j - a_jb_k)^2 \\
        = & \left( \sum_{k=1}^n a_k^2\right) \left( \sum_{k=1}^n b_k^2\right) - \sum_{1 \leq k < j \leq n}(a_k^2b_j^2 + a_j^2b_k^2 - 2a_ja_kb_jb_k)
            - \sum^n_{k=1}a_k^2b_k^2 + \sum^n_{k=1}a_k^2b_k^2 \\
        = & \left( \sum_{k=1}^n a_k^2\right) \left( \sum_{k=1}^n b_k^2\right) - \sum^n_{k=1} \sum^n_{j=1}a_k^2b_k^2 + \sum^n_{k=1}a_kb_k \sum^n_{j=1}a_kb_j \\
        = & \left( \sum^n_{k=1}a_kb_k \right)^2 \hfill \square
    \end{array}
\]
\I


\en

\I Let $F$ be an ordered field containing $\mathbb{Q}$.
Classify the following properties of $F$ to group A and group B
such that (1) properties in the same group are equivalent
to each other and (2) every property in group A
implies any property in group B, but not the converse.
\bn
\I The least-upper-bound property.
\I The archimedean property.
\I $\mathbb{Q}$ is dense in $F$.
\I For any $x\in F, x>0$, there exists $n\in \mathbb{N}$
such that $0<\frac 1 n< x$.
\I If $A$ and $B$ are nonempty
subsets of $F$ having the property that $x \le y$ for every 
$x \in A$ and every $y \in B$ , then
there exists $c \in F$ such that 
$x \le c \le y$ for all $x \in A$ and $y \in B$. 
 \en 
{\bf Ans.}
這裡展示,例如, $(a)\Rightarrow (b)$ 的寫法。
 \hfill $\square$

\I Let $Q[x]$ be the field
of rational functions with real coefficients.
For each $f\in Q[x]$ there exist unique
polynomials $p=\sum_{j=0}^n p_jx^j, 
q=\sum_{j=0}^m q_jx^j$ such that $p_n\ne 0, q_m=1$ and 
$f=p/q$ is in lowest terms (that is, $p$ and $q$ have no
nonconstant factors in common). With this notation, let
$K=\{ f \in Q[x]\,:\, p_n > 0\}$ and define
$f \prec g$ if $g-f \in K$.
\bn
\I Show that $\prec$ is an order which makes $Q[x]$ an 
ordered field.
\I Show that $Q[x]$ does not have the archimedean 
property. Give an upper bound and a lower bound to 
show that $\mathbb{N}$ is bounded in $Q[x]$.
\en
{\bf Ans.}

\I (Nested Interval Property)
A sequence of subsets $S_n$ is called {\it nested} if $S_{n+1}\subseteq
S_n, n\in \mathbb{N}$. 
Let $I_n=[a_n, b_n]\in \mathbb{R}$ be a nested sequence of 
closed intervals. 
\bn
\I
Prove that the nested sequence
has a nonempty intersection; 
that is, $\cap_{n=1}^\infty I_n\ne \emptyset$.
\I
Moreover, if it is known that for every $\varep>0$ there is an
interval $I_k$ whose length $|I_k| < \varep$, then the point
common to all the intervals is unique.
\I 
What can be said if the nested sets $I_n$ are open intervals
$I_n= (a_n, b_n), n\in \mathbb N$? Give a proof or an example
of your answer.
\en 
{\bf Ans.}
\en
\end{document}
