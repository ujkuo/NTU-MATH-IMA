\documentclass[a4pap er]{article}
%The main tex begins from line 188
\usepackage[top = 2.5cm, bottom = 2.5cm, left = 2cm, right = 2cm]{geometry} 
\usepackage{amsmath}
\usepackage{fancyhdr}
\usepackage{lastpage}
\usepackage{etoolbox}
\usepackage{indentfirst}
\usepackage{tabularx}
\usepackage{mathtools}
\usepackage{booktabs}
\usepackage{authblk}
\usepackage[calcwidth]{titlesec}
\usepackage{bookmark}
\usepackage[capitalize]{cleveref}
\usepackage{array}
\usepackage[english]{babel}
\usepackage{enumitem}
\usepackage{centernot}
%\usepackage[utf8]{inputenc}
%\usepackage{CJKutf8}
\usepackage{xeCJK}

%%%%%%%%%%%%%%%%%%%%%%%%%%%%%%%%%%%%%%%%%%%%%%%%%%%

\makeatletter
% Macros for setting basic info
\gdef\@uni{National Taiwan University}
\gdef\@department{Information Management}
\gdef\@me{Yu-Chieh Kuo}

\newcommand{\course}[2][]{
    \ifstrempty{#1}{
        \gdef\shortcourse{#2}
    }{
        \gdef\shortcourse{#1}
    }
    \gdef\@course{#2}
}
\newcommand{\teacher}[1]{\gdef\@teacher{#1}}
\newcommand{\semester}[1]{\gdef\@semester{#1}}
\newcommand{\uni}[1]{\gdef\@uni{#1}}
\newcommand{\department}[1]{\gdef\@department{#1}}
\newcommand{\student}[3][]{
    \ifstrempty{#1}{
        \author{\textbf{#2} \textbf{#3}}
    }{
        \author[#1]{\textbf{#2} \textbf{#3}}
    }
}
\newcommand{\assignment}[2][Homework]{
    \ifstrempty{#2}{
        \gdef\@assignment{#1}
    }{
        \gdef\@assignment{#1 #2}
    }
}

% page style for title page
\fancypagestyle{title}{
    \fancyhf{}
    \renewcommand{\headrulewidth}{0pt}
    \cfoot{\footnotesize Page {\thepage} of \pageref{LastPage}} 
}
\newcommand{\ujtitle}{
    \thispagestyle{title}
    \noindent\begin{tabularx}{\linewidth}{Xr} % This is a simple tabular environment to align your text nicely 
    {\large \bf{\@course}} \\
    National Taiwan University, \@semester  \\
    \bottomrule % \hline produces horizontal lines.
    \end{tabularx}
    
    \vspace*{0.3cm}

    \begin{center}
        {\huge\bf\textsc{\@assignment}}\\[3mm]
        {\@author}
    \end{center}

    \vspace*{0.3cm}
}
\newcommand{\mymaketitle}{
    \thispagestyle{title}
    \vspace*{-15mm}
    \noindent\begin{tabularx}{\linewidth}{Xr}
                                & \textsl{National Taiwan University} \\
        \large\textbf{\@semester,} & \textsl{Department of \@department}         \\
        \large\textbf{\@course} & \textsl{Prof. \@teacher}\Bstrut\\
        \bottomrule
    \end{tabularx}

    \vspace*{10mm}

    \begin{center}
        {\huge\textsc{\@assignment{}}}\\[6mm]
        {\@author}
    \end{center}

    \vspace*{1cm}
}

% page style for contents
\pagestyle{fancy}
\fancyhf{}
\setlength{\headheight}{13.6pt}
\lhead{\small {\@course}: {\@assignment}}
\rhead{\small {\@me}}
\cfoot{\small {Page {\thepage} of \pageref{LastPage}}} 

% Style for part, section, and subsection
% Part
\titleclass{\part}{straight}
\titleformat{\part}
[display]
{\centering\bfseries}
{\setlength{\titlewidth}{1.5\titlewidth}\large\titleline*[c]{\titlerule[.6pt]}\vspace{3pt}\MakeUppercase{\partname} \thepart}
{0pt}
{\LARGE\itshape}
[{
    \setlength{\titlewidth}{1.5\titlewidth}
    \titleline*[c]{\titlerule*{\mdseries-}}
}]
\titlespacing*{\part}{0pt}{20pt}{20pt}[0pt]

% Problem and subproblem
\setcounter{secnumdepth}{0}
\newcounter{problem}
\newcounter{subproblem}[problem]
\newbool{appendix}
\gdef\problemName{Problem}
\newcommand{\problem}[2][]{
    \setcounter{equation}{0}
    \ifstrempty{#1}{
        \stepcounter{problem}
    }{
        \setcounter{problem}{#1}
    }
    \ifstrempty{#2}{
        \ifbool{appendix}{
            \section{\hspace{-5mm}\problemName{} \Alph{problem}.}
        }{
            \section{\hspace{-5mm}\problemName{} \arabic{problem}}
        }
    }{
        \ifbool{appendix}{
            \section{\hspace{-5mm}\problemName{} \Alpa{problem}: #2}
        }{
            \section{\hspace{-5mm}\problemName{} \arabic{problem}: #2}
        }
    }
}
\newcommand{\subproblem}[2][]{
    \setcounter{equation}{0}
    \ifstrempty{#1}{
        \stepcounter{subproblem}
    }{
        \setcounter{subproblem}{#1}
    }
    \ifstrempty{#2}{
        \ifbool{appendix}{
            \section{\hspace{-2mm}\Alph{problem}.\arabic{subproblem}}
        }{
            \section{\hspace{-2mm}\arabic{problem}.(\alph{subproblem})}
        }
    }{
        \ifbool{appendix}{
            \section{\hspace{-2mm}\Alph{problem}.\arabic{subproblem}. #2}
        }{
            \section{ \hspace{-2mm}\arabic{problem}.(\alph{subproblem}) #2}
        }
    }
}

% Appendix
\newcommand{\myAppendix}{
    \setcounter{section}{0}
    \renewcommand{\sectionname}{Appendix}
    \renewcommand{\thesection}{\Alph{section}}
    \renewcommand{\thesubsection}{\Alph{section}.\arabic{subsection}}
    \renewcommand{\perhapscolon}[1]{\perhapsdot{##1}}
}

% Bookmarks
\hypersetup{
    bookmarksnumbered=true
}

% Reference nome for task and subtask
\crefname{section}{\sectionname}{\sectionnamepl}

\makeatother

%%%%%%%%%%%%%%%%%%%%%%%%%%%%%%%%%%%%%%%%%%%%%%%%%%%

% \everymath{\displaystyle}

% commands for spacing
\newcommand{\Tstrut}{\rule{0pt}{4mm}}         % = `top' strut
\newcommand{\Bstrut}{\rule[-2mm]{0mm}{0mm}}   % = `bottom' strut

% Paired Delimiters {}, (), []
\providecommand\given{} % so it exists
\newcommand\SetSymbol[1][]{
   \nonscript\,#1\vert \allowbreak \nonscript\,\mathopen{}}
\DeclarePairedDelimiterX\Set[1]{\lbrace}{\rbrace}%
 { \renewcommand\given{\SetSymbol[\delimsize]} #1 }
\DeclarePairedDelimiterX{\Bkt}[1]{[}{]}{#1}
\DeclarePairedDelimiterX{\Paren}[1]{(}{)}{#1}
\DeclarePairedDelimiterX{\Abs}[1]{\lvert}{\rvert}{#1}
\newcommand{\PR}[1]{\Pr\Bkt*{#1}}
\newcommand{\overbar}[1]{\mkern 1.5mu\overline{\mkern-1.5mu#1\mkern-1.5mu}\mkern 1.5mu}

\newcolumntype{D}{>{$(}l<{)$}}
\newcolumntype{R}{>{\displaystyle}r}
\newcolumntype{C}{>{\displaystyle}c}
\newcolumntype{L}{>{\displaystyle}l}

%%%%%%%%%%%%%%%%%%%%%%%%%%%%%%%%%%%%%%%%%%%%%%%%%%%

\usepackage{amsfonts}
\usepackage{amsthm}

%%%%%%%%%%%%%%%%%%%%%%%%%%%%%%%%%%%%%%%%%%%%%%%%%%%

% Declare operator and useful math command
\DeclareMathOperator{\EOp}{\mathbb{E}}
\newcommand{\E}[1]{\ensuremath{\EOp\Bkt*{#1}}}
\newcommand{\R}{\mathbb{R}}
\newcommand{\N}{\mathbb{N}}
\newcommand{\Q}{\mathbb{Q}}
\newcommand{\ie}{\text{ i.e., }}
\newcommand{\st}{\text{ s.t. }}
\newcommand{\ve}{\varepsilon}

% Declare delimiter
\DeclareMathDelimiter{\lVert}
  {\mathopen}{symbols}{"6B}{largesymbols}{"0D}
\DeclareMathDelimiter{\rVert}
  {\mathclose}{symbols}{"6B}{largesymbols}{"0D}
  \DeclarePairedDelimiter\norm{\lvert}{\rvert}

%%%%%%%%%%%%%%%%%%%%%%%%%%%%%%%%%%%%%%%%%%%%%%%%%%%

\theoremstyle{plain}
\newtheorem{corollary}{Corollary}
\newtheorem{proposition}{Proposition}
\newtheorem{lemma}{Lemma}

%%%%%%%%%%%%%%%%%%%%%%%%%%%%%%%%%%%%%%%%%%%%%%%%%%%

\setCJKmainfont{思源宋體 TW}
%\setmainfont{思源宋體 TW}

%%%%%%%%%%%%%%%%%%%%%%%%%%%%%%%%%%%%%%%%%%%%%%%%%%%

\begin{document}
%\begin{CJK*}{UTF8}{bsmi}
\course{Introduction to Mathematical Analysis (I)}
\semester{Fall 2021}
%\teacher{}
%\department{}
\assignment{1}
\date{\today}
\student[$\dagger$]{Yu-Chieh Kuo}{B07611039}
\affil[$\dagger$]{Department of Information Management, National Taiwan University}
\ujtitle
\problem{Definition and properties of exponential functions}
\subproblem{}
Since $b>1$, the function $t \mapsto b^t$ is increasing.
Given any $r, t \in \Q, \ t \leq r$, since $b^t \leq b^r$,
we have $\sup B(r) \leq b^r$.
Conversely, $r \in B(r) \implies b^r \leq \sup B(r)$. 
Therefore, it suffices to say $b^r = \sup B(r) \ \forall r \in \Q$.

\subproblem{}
Let 
\[
    \begin{array}{RCL}
        B(x) & = & \{ b^r | r \in \Q, \ r \leq x \} \\
        B(y) & = & \{ b^s | s \in \Q, \ s \leq y \} \\
        B(x+y) & = & \{ b^t | t \in \Q, \ t \leq x + y\} \\
               & = & \{b^rb^s | r, s \in \Q, \ r \leq x,
               \ s \leq y \},
    \end{array}
\]
and
\[
    \begin{array}{RCL}
        b^x & = & \sup B(x)\\
        b^y & = & \sup B(y) \\
        b^{x+y} & = & \sup B(x+y).
    \end{array}
\]
Since
\[
    B(x+y) = b^rb^s \leq b^xb^y = \sup B(x) \sup B(y),
\]
take supremun, $\sup B(x+y) \leq \sup B(x) \sup B(y)$.
Moreover, for $b^rb^s = b^{r+s} \leq \sup B(x+y)$, we have
\[
    \begin{array}{RL}
        & b^r \leq \frac{\sup B(x+y)}{b^s}  \\ [4mm]
        \implies & \sup B(x) \leq \frac{\sup B(x+y)}{b^s} \\[4mm]
        & \text{(Here $\frac{\sup B(x+y)}{b^s}$ is an
        upper bound of $b^r$.)} \\ [4mm]
        \implies & \sup B(x) \leq \frac{\sup B(x+y)}{\sup B(y)}  \\ [4mm]
        \implies & \sup B(x) \sup B(y) \leq \sup B(x+y). 
    \end{array}
\]
Therefore, we derive $\sup B(x) \sup B(y) = \sup B(x+y)$ 
i.e., $b^xb^y = b^{x+y}$ for all real $x$ and $y$.
\problem{Logarithm}
\begin{enumerate}[label = (\alph*)]
\item 
\[
    \begin{array}{LLL}
    b^n - 1 & = b^n - 1^ n & = (b-1)\underbrace{(b^{n-1}) + b^{n-2} + \cdots + 1}_{n} \\
        & & \geq (b-1)\underbrace{(1 + 1 + \cdots + 1)}_{n} \\
        & &= n(b-1).
    \end{array}
\]
\item Put $b \mapsto b^{\frac{1}{n}}$ in (a) then it's proven.
\item 
    \[
        n > \frac{b-1}{t-1} \implies n(t-1) > b-1 
        \geq n(b^{\frac{1}{n}} - 1),
    \]
    where $n,\, b,\, t$ are positive, hence, $t > b^{\frac{1}{n}}.$
\item
    Let $t = y\cdot b^{-w} > 1$. By (c), we have
\[
    \begin{array}{RLL}
        & b^{\frac{1}{n}} < y \cdot b^{-w} & \text{for } n > \frac{b-1}{y\cdot b^{-w}-1} \\
        \iff & b^{\frac{1}{n}+w} > y & \text{for } n > \frac{b-1}{y\cdot b^{-w}-1}.
    \end{array}
\]
\item
    Let $t = y^{-1}\cdot b^w > 1$. By (c), we have 
    \[
        \begin{array}{RLL}
            & b^{\frac{1}{n}} < y^{-1}\cdot b^w & \text{for } n > \frac{b-1}{y\cdot b^{-w}-1} \\
            \iff & b^{w-\frac{1}{n}} > y & \text{for } n > \frac{b-1}{y\cdot b^{-w}-1}.
        \end{array}
    \]
\item Suppose $b^x < y$, (d) implies that $b^{w+\frac{1}{n}}
    < y$ for sufficiently large $n$. 
    Suppose $b^x > y$, (e) implies that $b^{w - \frac{1}{n}}
    < y$ for sufficiently large $n$.
Both conditions above indicate that $x$ cannot be $\sup A$
satisfying $b^x = y$.
    Hence, $x = \sup A$ satisfies $b^x=y$.
\item
By contradiction, if there were another $x' \neq x$
such that $b^{x'} = y$, then $x'>x$ or $x'<x$.
For the case $x'>x$, $y=b^{x'} > b^x = y$, which is contradicted. 
Similarly, for the case $x' < x$, $y=b<{x'} < b^x=y$ is also
contradicted. Hence, $x$ should be unique.
\end{enumerate}

\problem{Cauchy-Schwarz Inequality}
\subproblem{} We start deriving from the left side.
\[
    \begin{array}{RL}
        & \left( \sum_{k=1}^n a_k^2\right) \left( \sum_{k=1}^n b_k^2\right)
        - \sum_{1 \leq k < j \leq n}(a_kb_j - a_jb_k)^2 \\
        = & \left( \sum_{k=1}^n a_k^2\right) \left( \sum_{k=1}^n b_k^2\right)
        - \sum_{1 \leq k < j \leq n}(a_k^2b_j^2 + a_j^2b_k^2 - 2a_ja_kb_jb_k)
            - \sum^n_{k=1}a_k^2b_k^2 + \sum^n_{k=1}a_k^2b_k^2 \\
        = & \left( \sum_{k=1}^n a_k^2\right) \left( \sum_{k=1}^n b_k^2\right)
        - \sum^n_{k=1} \sum^n_{j=1}a_k^2b_j^2 + \sum^n_{k=1}a_kb_k \sum^n_{j=1}a_kb_j \\
        = & \left( \sum_{k=1}^n a_k^2\right) \left( \sum_{k=1}^n b_k^2\right) 
        - \left( \sum_{k=1}^n a_k^2\right) \left( \sum_{k=1}^n b_k^2\right)
        + \sum^n_{k=1}a_kb_k \sum^n_{j=1}a_kb_j \\
        = & \left( \sum^n_{k=1}a_kb_k \right)^2.
    \end{array}
\]

\subproblem{}
\[
    \begin{array}{RCL}
        \lvert \langle{\bf a,b}\rangle\rvert^2 & = & \left|
        \sum_{k=1}^n a_k \bar{b}_k \right | ^2 \\
        & = & \sum_{k=1}^na_k \bar{b}_k \cdot  \sum_{k=1}^n\bar{a}_k b_k \\ [5mm]
        & = & \sum_{1 \leq j,k \leq n}a_k\bar{a}_j b_j\bar{b}_k \\ [5mm]
        & = & \sum_{k=1}^n\lvert a_k\bar{b}_k\rvert^2
        + \sum_{1\leq k < j \leq n}a_k\bar{a}_jb_j\bar{b}_k + a_j\bar{a}_kb_k\bar{b}_j
        + \sum_{1\leq j,k \leq n, j\neq k}\lvert a_k\bar{b}_j \rvert^2
        - \sum_{1\leq j,k \leq n, j\neq k}\lvert a_k\bar{b}_j \rvert^2 \\
        & = & \sum_{k=1}^n \lvert a_k\rvert^2 \sum_{j=1}^n \lvert \bar{b}_j\rvert^2
        - \left( \sum_{1\leq k < j \leq n} \lvert a_k\bar{b}_j\rvert^2
        + \lvert a_j\bar{b}_k\rvert^2
        + a_k\bar{a}_jb_j\bar{b}_k + a_j\bar{a}_kb_k\bar{b}_j \right) \\
        & = & \sum_{k=1}^n \lvert a_k\rvert^2 \sum_{j=1}^n \lvert \bar{b}_j\rvert^2
        - \sum_{1\leq k < j \leq n}\lvert a_k\bar{b}_j - a_j\bar{b}_k \rvert^2.
    \end{array}
\]
As a result, the Lagrange identity of complex number is
\[
    \left | \sum_{k=1}^n a_k \bar{b}_k \right | ^2 = 
    \sum_{k=1}^n \lvert a_k\rvert^2 \sum_{j=1}^n \lvert \bar{b}_j\rvert^2 -
    \sum_{1\leq k < j \leq n}\lvert a_k\bar{b}_j - a_j\bar{b}_k \rvert^2.
\]
\subproblem{}
Observe the results in 3(a) and 3(b) above, we may derive that
\[
    \begin{array}{RCL}
        \left( \sum_{k=1}^na_kb_k \right)^2 & \leq & 
        \left( \sum_{k=1}^na_k^2 \right) \left( \sum_{k=1}^nb_k^2 \right) \\
        \left | \sum_{k=1}^na_k\bar{b}_k\right |^2 & \leq &
        \sum_{k=1}^n \lvert a_k\rvert^2 \sum_{j=1}^n \lvert \bar{b}_j\rvert^2,
    \end{array}
\]
and these two inequalities representing the Cauchy-Schwarz inequalities
in respective real and complex number.
Under the condition that, when $\sum_{1\leq k<j\leq n}(a_kb_j-a_jb_k)^2$ and
$\sum_{1\leq k < j \leq n}\lvert a_k\bar{b}_j - a_j\bar{b}_k \rvert^2$ hold,
the inequalities hold respectively.

%p4
\problem{Equivalent Properties}
We classify that group $A$ contains (a), (e), and group $B$
contains (b), (c), (d).\\
\textbf{(Group A)}
\begin{enumerate}
    \item \textbf{(a) $\implies$ (e)} \\
Suppose order field $F$ holds the least-upper-bound property
(In the following, we abbreviate least-upper-bound property
as LUB property), then for two non-empty bounded below
subsets $A, B \subset F$, with the property that $x \leq y$
for every $x \in A$ and every $y \in B$, by \textbf{Theorem
1.11} in Rudin, we claim that $\inf B$ exists, and $\sup A$
exists as it's bounded by $B$.

Next, by the relation between $x$ and $y$, we derive
\[
    \begin{array}{RL}
        & x \leq y \ \forall x \in A, \ \forall y \in B \\
        \iff & \text{$y$ is an upper bound of $A$, } \forall
        y \in B \\
        \iff & y \geq \sup A, \ \forall y \in B \\
        \iff & \sup A \text{ is a lower bound of } B \\
        \iff & \sup A \leq \inf B.
    \end{array}
\]

Last, since $x \leq \sup A \leq \inf B \leq y, \ \forall x
\in A, \ \forall y \in B$, defince $c$ as $\sup A (\in F)$, 
then we prove that there exists $c \in F$ such that 
$x \leq c \leq y$ for all $x \in A$ and $y \in B$.

\item \textbf{(e) $\implies$ (a)} \\
Let $A, B$ be nomempty subsets of $F$ with the property 
that $x \leq y$ for every $x \in A$ and every $y \in B$,
and there exists $c \in F$ such that $x \leq c \leq y$ for
all $x \in A$ and $y \in B$. Since $c$ is an upper bound
of $A$ and $c \leq y$ i.e., c is bounded by any other 
upper bound. 
To expand to any set, consider $B'$ as the set of every upper bound,
and $A'$ as the set of those which are not upper bound.
For every element $x' \in A'$ and $y' \in B'$, we obtain the relation for
$x' \leq y'$. Since $A'$ is bounded above by $B'$, we could find $c' \in B'$
such that $x' \leq c' \leq y'$ for all $x' \in A'$ and $y' \in B'$.
Hence, $c'$ is the least upper bound, and we derive that if there exists the upper
bound of the set, then the LUB property is true.
\end{enumerate}
As above, e prove that properties in the group A are 
equivalent to each other. \\ [4mm]
\textbf{(Group B)}

\begin{enumerate}
    \item \textbf{(b) $\implies$ (c)} \\
Let $x, y \in F$ and $0 < y-x$. By the archimedean
property, there exists $n \in \N \st n(y-x) > 1$.
Moreover, there exists $m \in N \st nx < m < ny \implies
x < \frac{m}{n}$ and $y > \frac{m}{n}$. Therefore, we can
conclude that $\Q$ is in dense in $F$.
\item \textbf{(c) $\implies$ (d)} \\
    As $\Q$ is in dense in $F$, by (c), therefore, 
    for any $x \in F$, $x>0$, there exists
    $m,n \in \N \st 0 < \frac{1}{n} < \frac{m}{n} < x$

\item \textbf{(d) $\implies$ (b)} \\
    Let $x,y,z \in F, \ x,y,z > 0$ and $z = \frac{x}{y}$, 
    by (d), there exists $n \in N$ such that
    \[
    0 < \frac{1}{n} < z = \frac{x}{y}
    \implies nx > y.
    \]
    That is, (b) could be implied by (d).
\end{enumerate}
As above, we prove that properties in group B are equivalent
to each other.

Next, to prove that every property in group A implies any
properties in group B but not the converse, we take 
property (a) and property (b) as an example. Since the 
properties in the same group are equivalent, it suffices to
prove the condition between two groups by merely handling 
the relation between property (a) and (b).

\begin{enumerate}
\item \textbf{(a) $\implies$ (b)} \\
We use the fact that the set of positive integers $\N$
is not bounded above, to prove this deduction.
\begin{lemma} The set of positive integers $\N$ is not 
    bounded above.
\begin{proof}
    Since $\N \subset \Q \subset F$ and $F$ holds the LUB
    property i.e., $F$ has a least upper bound, 
    by contradiction, assume $\N$ has a least upper bound
    $a \in F$, where $n \leq a$ for all $n \in \N$.
    Consequently, $a-1$ is not an upper bound for $\N$
    as if it were, $a-1 < a$ implies that $a$ is not the
    \textit{least} upper bound. Therefore, there exists
    some integers $b$ such that $a-1 < b \implies a<b+1$.
    However, it contradicts that $a$ is an upper bound of
    $\N$.
\end{proof}
\end{lemma}
Let $x, y \in F, \ x > 0$ and $n \in \N$. Assume by
contradiction that there is no $n \in \N$ such that 
$nx > y$ i.e., the archimedean property is satisfied, then
it indicates $nx \leq y \iff n \leq \frac{y}{x}$ for all 
$n \in N$. This statement is equivalent that the set of 
positive integers $\N$ has an upper bound, and it 
contradicts the lemma. Therefore, we prove that the LUB
property implies the archimedean property.

\item \textbf{(b) $\centernot\implies$ (a)} \\
    Let $x \in \Q$ and $x > 0$, we can write $x$ as $\frac{n}{m}$ with $n \in \N$ and 
    $m \neq 0, \ m \in \N$. Since $m \geq 1$, then we have $n = mx \geq x$ which
    means $n+1 > x$. So the rational number has the archimedean property.
    However, $\Q$ does not have the LUB property. For example, consider
    $\{y \in \Q \  | y^2 = 2 \}$, which has an upper bound but no least upper bound.
    Hence, we find a counter-example to prove that property (b) cannot imply
    property (a).
\end{enumerate}

%p5
\problem{Ordered Field}
\subproblem{}
To indicate that $\prec$ is an order making $Q[x]$ an
ordered field, we need to prove
\begin{enumerate}
    \item $x+y \prec x+z$ if $x, y, z \in Q[x]$ and $y \prec z$.
    \item $0 \prec xy$ if $x, y \in Q[x]$, $0 \prec x$ and 
        $0 \prec y$.
\end{enumerate}
Let $x, y, z \in Q[x]$. For 1., assume that $x \prec y$, then
\[
    y - x \in K \implies y+z - (x+z) \in K \implies 
    x+z \prec y+z.
\]
For 2., assume that $0 \prec x$ and $0 \prec y$, it implies
\[
    x-0 \in K, \ y-0 \in K \implies x,y \in K.
\]
Since $x,y \in K$, their corresponding uniqle polynomials,
denote as $p^x, q^x, p^y, q^y$, satisfy that $p^x_n > 0, \ p^y_n > 0$ and $p^x_n\cdot p^y_n > 0 \ie xy \in K$. 
Hence, $0 \prec xy$ holds.
\subproblem{}
If $Q[x]$ have the archimedean property, then for 
$x, y \in Q[x]$ and $0 \prec x$, there exists a possitive 
number $n$ such that $y \prec nx \iff nx - y \in K$. 
However, when $m > n$, we could always find $y$ with
the positive leading coefficient such that
$nx - y \notin K$. Hence, $Q[x]$ does not have the 
archimedean property.

Let $f \in Q[x]$ with the positive leading coefficient
and $f = (n+1)x$ where $n \in \N$, then $\forall n \in \N,
\ \forall x>0,$
\[
    nx < f = nx+n \implies f \text{ is an upper bound of }nx.
\]
Similarly, let $g \in Q[x]$ with the positive leading 
coefficient and $g = (n-1)x$ where $n \in \N$, then 
$\forall n \in \N, \ \forall x > 0$,
\[
    nx > g = nx-x \implies g \text{ is a lower bound of }nx.
\]
Hence, it suffices to say that $\N$ is bounded in $Q[x]$.

%p6
\problem{Nested Interval Property}
\subproblem{}
Since $I_n=[a_n, b_n]$ is a nested sequence of closed intervals, we obtain that
\[
    \forall n,m \in \N, \ a_n \leq a_{\max \{n,m\}} \leq b_{\max \{n,m\}} \leq b_n.
\]
In other words, for every $m \in \N$, $b_m$ is a upper bound of $a_n \ (n \in \N)$.
Let $c = \lim\limits_{n\to \infty}a_n$, and $b_m \ (m \in \N)$ is an upper bound
of $a_n \ (n \in \N)$, we have $c \leq b_m \ \forall m \in \N$;
also, $a_n \leq c \ \forall n \in \N$. Therefore, we find $c \in \cap_{n=1}^\infty I_n
\ \ie \cap_{n=1}^\infty I_n \neq \emptyset$.

\subproblem{}
Let $A = \{a_n\}(n \in \N)$ and $B = \{b_n\}(n \in N)$, 
$a_n$ is a strictly increasing sequence, $b_n$ is a strictly
decreasing sequence, and $b_n(n \in \N)$ is an upper bound
of $A$. Since $a_n$ and $b_n$ are bounded, $a_n$ and $b_n$ 
converge to some limit points $\alpha, \beta$ respectively,
and $\alpha \leq \beta$. 
For any arbitrarily small $\varepsilon > 0$ and 
$\lvert I_k \rvert = b_k - a_k$, we have 
\[
    \lim_{k\to \infty} \lvert I_k \rvert
    = \lim_{k\to \infty} b_k - a_k
    = \lim_{k\to \infty}b_k - \lim_{k\to \infty}a_k
    = \beta - \alpha = 0
    \implies \alpha = \beta.
\]
Therefore, the point common to all the intervals
$\cap_{n=1}^\infty I_n$ is unique.

\subproblem{}
If the nested sets $I_n$ are open intervals, their intersection may sometimes be
an emptyset. For example, let the nested set $I_n = (0, \frac{1}{n})$, the intersection
of $I_n$ is an emptyset.

%\end{CJK*}
\end{document}

