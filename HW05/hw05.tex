\documentclass[a4paper]{article}
%The main tex begins from line 188
\usepackage[top = 2.5cm, bottom = 2.5cm, left = 2cm, right = 2cm]{geometry} 
\usepackage{amsmath} % For the usage of \because and \therefore
\usepackage{amssymb}
\usepackage{fancyhdr}
\usepackage{lastpage}
\usepackage{etoolbox}
\usepackage{indentfirst}
\usepackage{tabularx}
\usepackage{mathtools}
\usepackage{booktabs}
\usepackage{authblk}
\usepackage[calcwidth]{titlesec}
\usepackage{bookmark}
\usepackage[capitalize]{cleveref}
\usepackage{array}
\usepackage[english]{babel}
\usepackage{enumitem} % For the usage of labeling in enumerate and itemize
%\usepackage[utf8]{inputenc}
%\usepackage{CJKutf8}
\usepackage{xeCJK}

%%%%%%%%%%%%%%%%%%%%%%%%%%%%%%%%%%%%%%%%%%%%%%%%%%%

\makeatletter
% Macros for setting basic info
\gdef\@uni{National Taiwan University}
\gdef\@department{Information Management}
\gdef\@me{Yu-Chieh Kuo}

\newcommand{\course}[2][]{
    \ifstrempty{#1}{
        \gdef\shortcourse{#2}
    }{
        \gdef\shortcourse{#1}
    }
    \gdef\@course{#2}
}
\newcommand{\teacher}[1]{\gdef\@teacher{#1}}
\newcommand{\semester}[1]{\gdef\@semester{#1}}
\newcommand{\uni}[1]{\gdef\@uni{#1}}
\newcommand{\department}[1]{\gdef\@department{#1}}
\newcommand{\student}[3][]{
    \ifstrempty{#1}{
        \author{\textbf{#2} \textbf{#3}}
    }{
        \author[#1]{\textbf{#2} \textbf{#3}}
    }
}
\newcommand{\assignment}[2][Homework]{
    \ifstrempty{#2}{
        \gdef\@assignment{#1}
    }{
        \gdef\@assignment{#1 #2}
    }
}

% page style for title page
\fancypagestyle{title}{
    \fancyhf{}
    \renewcommand{\headrulewidth}{0pt}
    \cfoot{\footnotesize Page {\thepage} of \pageref{LastPage}} 
}
\newcommand{\ujtitle}{
    \thispagestyle{title}
    \noindent\begin{tabularx}{\linewidth}{Xr} % This is a simple tabular environment to align your text nicely 
    {\large \bf{\@course}} \\
    National Taiwan University, \@semester  \\
    \bottomrule % \hline produces horizontal lines.
    \end{tabularx}
    
    \vspace*{0.3cm}

    \begin{center}
        {\huge\bf\textsc{\@assignment}}\\[3mm]
        {\@author}
    \end{center}

    \vspace*{0.3cm}
}
\newcommand{\mymaketitle}{
    \thispagestyle{title}
    \vspace*{-15mm}
    \noindent\begin{tabularx}{\linewidth}{Xr}
                                & \textsl{National Taiwan University} \\
        \large\textbf{\@semester,} & \textsl{Department of \@department}         \\
        \large\textbf{\@course} & \textsl{Prof. \@teacher}\Bstrut\\
        \bottomrule
    \end{tabularx}

    \vspace*{10mm}

    \begin{center}
        {\huge\textsc{\@assignment{}}}\\[6mm]
        {\@author}
    \end{center}

    \vspace*{1cm}
}

% page style for contents
\pagestyle{fancy}
\fancyhf{}
\setlength{\headheight}{13.6pt}
\lhead{\small {\@course}: {\@assignment}}
\rhead{\small {\@me}}
\cfoot{\small {Page {\thepage} of \pageref{LastPage}}} 

% Style for part, section, and subsection
% Part
\titleclass{\part}{straight}
\titleformat{\part}
[display]
{\centering\bfseries}
{\setlength{\titlewidth}{1.5\titlewidth}\large\titleline*[c]{\titlerule[.6pt]}\vspace{3pt}\MakeUppercase{\partname} \thepart}
{0pt}
{\LARGE\itshape}
[{
    \setlength{\titlewidth}{1.5\titlewidth}
    \titleline*[c]{\titlerule*{\mdseries-}}
}]
\titlespacing*{\part}{0pt}{20pt}{20pt}[0pt]

% Problem and subproblem
\setcounter{secnumdepth}{0}
\newcounter{problem}
\newcounter{subproblem}[problem]
\newbool{appendix}
\gdef\problemName{Problem}
\newcommand{\problem}[2][]{
    \setcounter{equation}{0}
    \ifstrempty{#1}{
        \stepcounter{problem}
    }{
        \setcounter{problem}{#1}
    }
    \ifstrempty{#2}{
        \ifbool{appendix}{
            \section{\hspace{-5mm}\problemName{} \Alph{problem}.}
        }{
            \section{\hspace{-5mm}\problemName{} \arabic{problem}}
        }
    }{
        \ifbool{appendix}{
            \section{\hspace{-5mm}\problemName{} \Alpa{problem}: #2}
        }{
            \section{\hspace{-5mm}\problemName{} \arabic{problem}: #2}
        }
    }
}
\newcommand{\subproblem}[2][]{
    \setcounter{equation}{0}
    \ifstrempty{#1}{
        \stepcounter{subproblem}
    }{
        \setcounter{subproblem}{#1}
    }
    \ifstrempty{#2}{
        \ifbool{appendix}{
            \section{\hspace{-2mm}\Alph{problem}.\arabic{subproblem}}
        }{
            \section{\hspace{-2mm}\arabic{problem}.(\alph{subproblem})}
        }
    }{
        \ifbool{appendix}{
            \section{\hspace{-2mm}\Alph{problem}.\arabic{subproblem}. #2}
        }{
            \section{\hspace{-2mm}\arabic{problem}.(\alph{subproblem}) #2}
        }
    }
}

% Appendix
\newcommand{\myAppendix}{
    \setcounter{section}{0}
    \renewcommand{\sectionname}{Appendix}
    \renewcommand{\thesection}{\Alph{section}}
    \renewcommand{\thesubsection}{\Alph{section}.\arabic{subsection}}
    \renewcommand{\perhapscolon}[1]{\perhapsdot{##1}}
}

% Bookmarks
\hypersetup{
    bookmarksnumbered=true
}

% Reference nome for task and subtask
\crefname{section}{\sectionname}{\sectionnamepl}

\makeatother

%%%%%%%%%%%%%%%%%%%%%%%%%%%%%%%%%%%%%%%%%%%%%%%%%%%

% \everymath{\displaystyle}

% commands for spacing
\newcommand{\Tstrut}{\rule{0pt}{4mm}}         % = `top' strut
\newcommand{\Bstrut}{\rule[-2mm]{0mm}{0mm}}   % = `bottom' strut

% Paired Delimiters {}, (), []
\providecommand\given{} % so it exists
\newcommand\SetSymbol[1][]{
   \nonscript\,#1\vert \allowbreak \nonscript\,\mathopen{}}
\DeclarePairedDelimiterX\Set[1]{\lbrace}{\rbrace}%
 { \renewcommand\given{\SetSymbol[\delimsize]} #1 }
\DeclarePairedDelimiterX{\Bkt}[1]{[}{]}{#1}
\DeclarePairedDelimiterX{\Paren}[1]{(}{)}{#1}
\DeclarePairedDelimiterX{\Abs}[1]{\lvert}{\rvert}{#1}
\newcommand{\PR}[1]{\Pr\Bkt*{#1}}
\newcommand{\overbar}[1]{\mkern 1.5mu\overline{\mkern-1.5mu#1\mkern-1.5mu}\mkern 1.5mu}

\newcolumntype{D}{>{$(}l<{)$}}
\newcolumntype{R}{>{\displaystyle}r}
\newcolumntype{C}{>{\displaystyle}c}
\newcolumntype{L}{>{\displaystyle}l}

%%%%%%%%%%%%%%%%%%%%%%%%%%%%%%%%%%%%%%%%%%%%%%%%%%%

\usepackage{amsfonts}
\usepackage{amsthm}

%%%%%%%%%%%%%%%%%%%%%%%%%%%%%%%%%%%%%%%%%%%%%%%%%%%

% Declare operator and useful math command
\DeclareMathOperator{\EOp}{\mathbb{E}}
\newcommand{\E}[1]{\ensuremath{\EOp\Bkt*{#1}}}
\newcommand{\R}{\mathbb{R}}
\newcommand{\N}{\mathbb{N}}
\newcommand{\Q}{\mathbb{Q}}
\newcommand{\Z}{\mathbb{Z}}
\newcommand{\PS}[1]{\mathcal{P}(#1)}
\newcommand{\ve}{\varepsilon}

\newcommand{\sst}{\subset}
\newcommand{\sse}{\subseteq}

\newcommand{\ie}{\text{ i.e., }}
\newcommand{\st}{\text{ s.t.  }}


% Declare delimiter
\DeclareMathDelimiter{\lVert}
  {\mathopen}{symbols}{"6B}{largesymbols}{"0D}
\DeclareMathDelimiter{\rVert}
  {\mathclose}{symbols}{"6B}{largesymbols}{"0D}
\DeclarePairedDelimiter\norm{\lVert}{\rVert}

%%%%%%%%%%%%%%%%%%%%%%%%%%%%%%%%%%%%%%%%%%%%%%%%%%%

\theoremstyle{plain}
\newtheorem{corollary}{Corollary}
\newtheorem{proposition}{Proposition}

%%%%%%%%%%%%%%%%%%%%%%%%%%%%%%%%%%%%%%%%%%%%%%%%%%%

\setCJKmainfont{思源宋體 TW}
%\setmainfont{思源宋體 TW}

%%%%%%%%%%%%%%%%%%%%%%%%%%%%%%%%%%%%%%%%%%%%%%%%%%%

\begin{document}
%\begin{CJK*}{UTF8}{bsmi}
\course{Introduction to Mathematical Analysis (I)}
\semester{Fall 2021}
%\teacher{}
\department{}
\assignment{5}
\date{\today}
\student[$\dagger$]{Yu-Chieh Kuo}{B07611039}
\affil[$\dagger$]{Department of Information Management, National Taiwan University}
\ujtitle
\problem{}
We can prove the property from both sides respectively. Let $S$ be a topological space.
\begin{itemize}
    \item {($\Rightarrow$)} \\
    Suppose $S$ is connected. Based on the
    definition of set,
    the only two open and closed subsets
    are $\emptyset$ and the set $S$.
    Hence, it's impossible to find a subset $T$ satisfying that $T$ is nonempty,
    $T$ is a subset of $S$, and $T$ is 
    an open and closed subset in $S$.
    \item ($\Leftarrow$) \\
    Let $S$ be disconnected, therefore there are two nonempty {\bf open} subsets $T, U \subseteq S$,
    where $S = T \cup U$ and $T \cap U = \emptyset$.
    In addition, since $T^c = U$ and $U^c = T$, $T$ and $U$ are open and closed.
    Hence, if there doesn't exist such subsets as same as $T, U \subseteq S$ satisfying
    $S = T \cup U$ and $T \cap U = \emptyset$, we claim that $S$ is connected.
\end{itemize}
Consequently, we can say that a topological space $S$ is connected if and only if
the only subsets of $S$ which are both open and closed in $S$ are the empty set $\emptyset$
and $S$.

% p2
\problem{}
Interiors of connected sets are not always connected. Consider $X \subseteq \R^2$ be a metric space
with the normal Euclidean metric. Define $B_r(x)$ be an open ball centering at $x$ in radius $r$. Take
\[
    E = B_1((2,0)) \cup B_1((100,0)) \cup \{(x,0) \in \R^2 : 2 \leq x \leq 100 \}.
\]
Here $E$ is connected but 
\[
    E^\circ = B_1((2,0)) \cup B_1((100,0))
\]
is disconnected.

Closures of connected sets are always connected. It suffices to show that $E$ is disconnected if $\bar{E}$
is disconnected. Write $\bar{E} = A \cup B$ as an union of two nonempty separated sets, and
$A \cap \bar{B} = \emptyset, \ \bar{A} \cap B = \emptyset$. Write $E$ as $E = (A \cap E) \cup (B \cap E)$,
and it shows that $E$ is disconnected.
Next, we would like to show that $A \cap E$ and $B \cap E$ are separated.
In fact,
\[
    (A \cap E) \cup \overline{B \cap E} \subseteq A \cap \bar{B} = \emptyset,
\]
\[
    \overline{A \cap E} \cap (B \cap E) \subseteq \bar{A} \cap B = \emptyset.
\]
If $A \cap E = \emptyset$, therefore
\[
    E = (A \cap E) \cup (B \cap E) = B \cap E \implies E \subseteq B,
\]
and 
\[
    \begin{array}{RCL}
        A & = & (A \cup B) \cap A \\
        & = & \bar{E} \cap A \\
        & \subseteq & \bar{B} \cap A \\
        & = & \emptyset,
    \end{array}
\]
contrary to the definition that $A$ is nonempty. Hence, $A \cap E \neq \emptyset$ here, and
$B \cap E \neq \emptyset$ similarly. Consequently, $E$ is disconnected if $\bar{E}$ is disconnected,
or closures of connected sets are always connected.

% p3
\problem{}
\subproblem{}
Note that ${\bf a} \neq {\bf b}$ or $\lvert \bf{ a - b} \rvert > 0$ since
$A \cup B = \emptyset$, and $|{\bf p}(t) - {\bf p}(s)| = |t-s||{\bf a-b}|$. Hence, ${\bf p}(t) = {\bf p}(s)$
if and only if $t=s$. Now we want to show that $A_0$ and $B_0$ are separated i.e., $A_0 \cap \overline{B_0} = \emptyset$
and $\overline{A_0} \cap B_0 = \emptyset$.
If there is $t \in A_0 \cap \overline{B_0}$, then $t \in A_0$ and $t$ is also
a limit point of $B_0$. 
Since $t \in A_0$, ${\bf p}(t) \in A$ as well. In addition, given any $\ve > 0$,
there is $s \in B_0$ such that $|t-s| < \frac{\ve}{|{\bf a-b}|}$ for any $s \neq t$, which implies that
${\bf p}(t) - {\bf p}(s) = |t-s||{\bf a-b}| < \ve$. Here ${\bf p}(s) \in B$ and ${\bf p}(s) \neq {\bf p}(t)$, so 
${\bf p}(t)$ is a limit point of $B$. Therefore, ${\bf p}(t) \in A \cap \bar{B} = \emptyset$, contrary to the assumption that
$A$ and $B$ are separated. It's similar to show that $\overline{A_0} \cap B_0 = \emptyset$. Therefore, we prove that 
$A_0$ and $B_0$ are separated.

\subproblem{}
Assume for contradiction that for all $t \in (0,1)$, ${\bf p}(t)\in A\cup B$. 
Since ${\bf p}(0)={\bf a}\in A$ and ${\bf p}(1)={\bf b}\in B$, it follows that 
${\bf p}(t)\in A\cup B, \forall t\in [0,1]$. By definition of $A_0, B_0$, we have $[0,1]\subseteq A_0\cup B_0$. 
Let $U=[0,1]\cap A_0$ and $V=[0,1]\cap B_0$. Since
\[
    {\bf p}(0)={\bf a}\in A\implies 0\in A_0\implies 0\in U,
\]
\[
    {\bf p}(1)={\bf b}\in B\implies 1\in B_0\implies 1\in V,
\]
$U,V$ are nonempty here.
Since $A_0, B_0$ are separated according to {\bf 3.(a)}, it shows that $U, V$ are separated and hence $[0,1]$ is separated,
which leads to a contradiction.

\subproblem{}
Let $S$ be a subset of $\R^k$. $S$ is convex if 
$\lambda {\bf x}+(1-\lambda) {\bf y}\in S, \forall {\bf x, y}\in S$ and $\lambda \in (0,1)$. 
Assume for contradiction that $S$ is separated, then there exist
nonempty sets $X,Y\st X\cup Y=S$ and $X\cap Y=\emptyset$. Pick ${\bf x}\in X$ and ${\bf y}\in Y$, by {\bf 3.(b)}, 
there exists $\lambda\in (0,1)$ such that $\lambda {\bf x}+(1-\lambda) {\bf y}\notin S$, which contradicts the assumption that $S$ is a convex set.

% p4
\problem{}
Following by the {\bf Definition 3.16} in Rudin, we now put $s^* = \sup E$ and $s_* = \inf E$,
where the set $E$ contains all subsequential limits as defined in {\bf Definition 3.5} in Rudin,
and plus possibly the number $+\infty, -\infty$.
Now we attempt to show that the $\limsup$ and $\liminf$ defined in the statement are the same as
that given in Rudin.
\begin{itemize}
    \item $\limsup_{n\to\infty}x_n = s^*$: \\
    If $(x_n)$ is not bounded above, $\limsup_{n\to\infty}x_n = +\infty = s^*$.
    
    If $(x_n)$ is bounded above, for all $\ve > 0$ there exists such $N \in \N$ such that
    \[
        \limsup_{n\to\infty} x_n - \ve < x_N < \lim_{n\to\infty}\left(\sup_{k\geq n}x_k\right)
        := \limsup_{n\to\infty}x_n.
    \]
    Therefore $\limsup_{n\to\infty}x_n := \lim_{n\to\infty}\left(\sup_{k\geq n}x_k\right)$
    is a limit point of some subsequence of $(x_n)$, and it suffices to say $\limsup_{n\to\infty}x_n \in E$.
    
    Next, we would like to show that $\limsup_{n\to\infty}x_n$ is bounded.
    Since $\limsup_{n\to\infty}x_n \in E$, assume that there exists $m > \limsup_{n\to\infty}x_n$, $m \in \R$, such that 
    for some $n$, $x_n \geq m$ i.e.,
    \[
        \lim_{n'\to\infty}x_{n'} > \lim_{n\to\infty}\sup_{k\geq n}x_n,
    \]
    which leads to a contradiction of $\sup$. Therefore, if for some $m > \limsup_{n\to\infty}x_n$, $m \in \R$,
    $x_n < m$ for $n \geq N$, where $N \in \N$.
    
    By {\bf Definition 3.17} in Rudin, $s^*$ is the only number with the properties above. Hence,
    $\limsup_{n\to\infty}x_n$ defined in the statement is the same as $s^*$.
    \item $\liminf_{n\to\infty}x_n = s_*$: \\
    Following the same concept to prove $\limsup_{n\to\infty}x_n = s^*$.
    If $(x_n)$ is not bounded below, $\liminf_{n\to\infty}x_n = -\infty = s_*$.
    
    If $(x_n)$ is bounded below, for all $\ve > 0$ there exists such $N \in \N$ such that
    \[
        \liminf_{n\to\infty} x_n := \lim_{n\to\infty}\left(\inf_{k\geq n}x_k\right) < x_N < \liminf_{n\to\infty}x_n + \ve.
    \]
    Therefore $\liminf_{n\to\infty}x_n := \lim_{n\to\infty}\left(\inf_{k\geq n}x_k\right)$
    is a limit point of some subsequence of $(x_n)$, and it suffices to say $\liminf_{n\to\infty}x_n \in E$.
    
    Next, we would like to show that $\liminf_{n\to\infty}x_n$ is bounded.
    Since $\liminf_{n\to\infty}x_n \in E$, assume that there exists $m > \liminf_{n\to\infty}x_n$, $m \in \R$, such that 
    for some $n$, $x_n \geq m$ i.e.,
    \[
        \lim_{n'\to\infty}x_{n'} < \lim_{n\to\infty}\inf_{k\geq n}x_n,
    \]
    which leads to a contradiction of $\inf$. Therefore, if for some $m < \liminf_{n\to\infty}x_n$, $m \in \R$,
    $x_n > m$ for $n \geq N$, where $N \in \N$.
    
    By {\bf Definition 3.17} in Rudin, $s_*$ is the only number with the properties above. Hence,
    $\liminf_{n\to\infty}x_n$ defined in the statement is the same as $s_*$.
\end{itemize}

% p5
\problem{}
Following the concept in Rudin for {\bf Theorem 3.37}, we put $\alpha = \liminf_{n\to\infty}\frac{c_{n+1}}{c_n}$.
If $\alpha = 0$, it's nothing to prove. If $\alpha > 0$, choose $\beta > \alpha$, and there is an integer $N$ such that
$\beta \leq \frac{c_{n+1}}{c_n}$ as $n \geq N$. In particular, for any $p > 0, p \in \N$, we have
\[
    \beta \cdot c_{N+k} \leq c_{N+k+1} \quad (k = 0,1,\dots,p-1).
\]
Multiplying these inequalities, we obtain
\[
    \beta^p c_{N} \leq c_{N+p} \quad \text{or} \quad c_N\beta^{-N}\cdot\beta^n \leq c_n
\]
for $n \geq N$. Hence, it shows
\[
    \beta\cdot\sqrt[n]{\beta^{-N}c_{N}} \leq \sqrt[n]{c_{n}},
\]
so that $\liminf_{n\to\infty}\sqrt[n]{c_n}\geq\beta$. Therefore, by {\bf Theorem 3.20} in Rudin, since 
$\beta\cdot\sqrt[n]{\beta^{-N}c_{N}} \leq \sqrt[n]{c_{n}}$ is true for every $\beta > \alpha$, we prove the inequality.

%\end{CJK*}
\end{document}