\documentclass[a4paper]{article}
%The main tex begins from line 262
\usepackage[top = 2.5cm, bottom = 2.5cm, left = 2cm, right = 2cm]{geometry} 
\usepackage{amsmath} % For the usage of \because and \therefore
\usepackage{amssymb}
\usepackage{fancyhdr}
\usepackage{lastpage}
\usepackage{etoolbox}
\usepackage{indentfirst}
\usepackage{tabularx}
\usepackage{mathtools}
\usepackage{booktabs}
\usepackage{authblk}
\usepackage[calcwidth]{titlesec}
\usepackage{bookmark}
\usepackage[capitalize]{cleveref}
\usepackage{array}
\usepackage[english]{babel}
\usepackage{enumitem} % For the usage of labeling in enumerate and itemize
%\usepackage[utf8]{inputenc}
%\usepackage{CJKutf8}
\usepackage{xeCJK}

%%%%%%%%%%%%%%%%%%%%%%%%%%%%%%%%%%%%%%%%%%%%%%%%%%%

\makeatletter
% Macros for setting basic info
\gdef\@uni{National Taiwan University}
\gdef\@department{Information Management}
\gdef\@me{Yu-Chieh Kuo}

\newcommand{\course}[2][]{
    \ifstrempty{#1}{
        \gdef\shortcourse{#2}
    }{
        \gdef\shortcourse{#1}
    }
    \gdef\@course{#2}
}
\newcommand{\teacher}[1]{\gdef\@teacher{#1}}
\newcommand{\semester}[1]{\gdef\@semester{#1}}
\newcommand{\uni}[1]{\gdef\@uni{#1}}
\newcommand{\department}[1]{\gdef\@department{#1}}
\newcommand{\student}[3][]{
    \ifstrempty{#1}{
        \author{\textbf{#2} \textbf{#3}}
    }{
        \author[#1]{\textbf{#2} \textbf{#3}}
    }
}
\newcommand{\assignment}[2][Homework]{
    \ifstrempty{#2}{
        \gdef\@assignment{#1}
    }{
        \gdef\@assignment{#1 #2}
    }
}

% page style for title page
\fancypagestyle{title}{
    \fancyhf{}
    \renewcommand{\headrulewidth}{0pt}
    \cfoot{\footnotesize Page {\thepage} of \pageref{LastPage}} 
}
\newcommand{\ujtitle}{
    \thispagestyle{title}
    \noindent\begin{tabularx}{\linewidth}{Xr} % This is a simple tabular environment to align your text nicely 
    {\large \bf{\@course}} \\
    National Taiwan University, \@semester  \\
    \bottomrule % \hline produces horizontal lines.
    \end{tabularx}
    
    \vspace*{0.3cm}

    \begin{center}
        {\huge\bf\textsc{\@assignment}}\\[3mm]
        {\@author}
    \end{center}

    \vspace*{0.3cm}
}
\newcommand{\mymaketitle}{
    \thispagestyle{title}
    \vspace*{-15mm}
    \noindent\begin{tabularx}{\linewidth}{Xr}
                                & \textsl{National Taiwan University} \\
        \large\textbf{\@semester,} & \textsl{Department of \@department}         \\
        \large\textbf{\@course} & \textsl{Prof. \@teacher}\Bstrut\\
        \bottomrule
    \end{tabularx}

    \vspace*{10mm}

    \begin{center}
        {\huge\textsc{\@assignment{}}}\\[6mm]
        {\@author}
    \end{center}

    \vspace*{1cm}
}

% page style for contents
\pagestyle{fancy}
\fancyhf{}
\setlength{\headheight}{13.6pt}
\lhead{\small {\@course}: {\@assignment}}
\rhead{\small {\@me}}
\cfoot{\small {Page {\thepage} of \pageref{LastPage}}} 

% Style for part, section, and subsection
% Part
\titleclass{\part}{straight}
\titleformat{\part}
[display]
{\centering\bfseries}
{\setlength{\titlewidth}{1.5\titlewidth}\large\titleline*[c]{\titlerule[.6pt]}\vspace{3pt}\MakeUppercase{\partname} \thepart}
{0pt}
{\LARGE\itshape}
[{
    \setlength{\titlewidth}{1.5\titlewidth}
    \titleline*[c]{\titlerule*{\mdseries-}}
}]
\titlespacing*{\part}{0pt}{20pt}{20pt}[0pt]

% Problem and subproblem
\setcounter{secnumdepth}{0}
\newcounter{problem}
\newcounter{subproblem}[problem]
\newbool{appendix}
\gdef\problemName{Problem}
\newcommand{\problem}[2][]{
    \setcounter{equation}{0}
    \ifstrempty{#1}{
        \stepcounter{problem}
    }{
        \setcounter{problem}{#1}
    }
    \ifstrempty{#2}{
        \ifbool{appendix}{
            \section{\hspace{-5mm}\problemName{} \Alph{problem}.}
        }{
            \section{\hspace{-5mm}\problemName{} \arabic{problem}}
        }
    }{
        \ifbool{appendix}{
            \section{\hspace{-5mm}\problemName{} \Alpa{problem}: #2}
        }{
            \section{\hspace{-5mm}\problemName{} \arabic{problem}: #2}
        }
    }
}
\newcommand{\subproblem}[2][]{
    \setcounter{equation}{0}
    \ifstrempty{#1}{
        \stepcounter{subproblem}
    }{
        \setcounter{subproblem}{#1}
    }
    \ifstrempty{#2}{
        \ifbool{appendix}{
            \section{\hspace{-2mm}\Alph{problem}.\arabic{subproblem}}
        }{
            \section{\hspace{-2mm}\arabic{problem}.(\alph{subproblem})}
        }
    }{
        \ifbool{appendix}{
            \section{\hspace{-2mm}\Alph{problem}.\arabic{subproblem}. #2}
        }{
            \section{\hspace{-2mm}\arabic{problem}.(\alph{subproblem}) #2}
        }
    }
}

% Appendix
\newcommand{\myAppendix}{
    \setcounter{section}{0}
    \renewcommand{\sectionname}{Appendix}
    \renewcommand{\thesection}{\Alph{section}}
    \renewcommand{\thesubsection}{\Alph{section}.\arabic{subsection}}
    \renewcommand{\perhapscolon}[1]{\perhapsdot{##1}}
}

% Bookmarks
\hypersetup{
    bookmarksnumbered=true
}

% Reference nome for task and subtask
\crefname{section}{\sectionname}{\sectionnamepl}

\makeatother

%%%%%%%%%%%%%%%%%%%%%%%%%%%%%%%%%%%%%%%%%%%%%%%%%%%

\everymath{\displaystyle} % display all math symbol as DISPLAY MODE

% commands for spacing
\newcommand{\Tstrut}{\rule{0pt}{4mm}}         % = `top' strut
\newcommand{\Bstrut}{\rule[-2mm]{0mm}{0mm}}   % = `bottom' strut

% Paired Delimiters {}, (), []
\providecommand\given{} % so it exists
\newcommand\SetSymbol[1][]{
   \nonscript\,#1\vert \allowbreak \nonscript\,\mathopen{}}
\DeclarePairedDelimiterX\Set[1]{\lbrace}{\rbrace}%
 { \renewcommand\given{\SetSymbol[\delimsize]} #1 }
\DeclarePairedDelimiterX{\Bkt}[1]{[}{]}{#1}
\DeclarePairedDelimiterX{\Paren}[1]{(}{)}{#1}
\DeclarePairedDelimiterX{\Abs}[1]{\lvert}{\rvert}{#1}
\newcommand{\PR}[1]{\Pr\Bkt*{#1}}
\newcommand{\overbar}[1]{\mkern 1.5mu\overline{\mkern-1.5mu#1\mkern-1.5mu}\mkern 1.5mu}

\newcolumntype{D}{>{$(}l<{)$}}
\newcolumntype{R}{>{\displaystyle}r}
\newcolumntype{C}{>{\displaystyle}c}
\newcolumntype{L}{>{\displaystyle}l}

%%%%%%%%%%%%%%%%%%%%%%%%%%%%%%%%%%%%%%%%%%%%%%%%%%%

\usepackage{amsfonts}
\usepackage{amsthm}

%%%%%%%%%%%%%%%%%%%%%%%%%%%%%%%%%%%%%%%%%%%%%%%%%%%

% Declare operator and useful math command
\DeclareMathOperator{\EOp}{\mathbb{E}}
\newcommand{\E}[1]{\ensuremath{\EOp\Bkt*{#1}}}
\newcommand{\R}{\mathbb{R}}
\newcommand{\N}{\mathbb{N}}
\newcommand{\Q}{\mathbb{Q}}
\newcommand{\Z}{\mathbb{Z}}
\newcommand{\PS}[1]{\mathcal{P}(#1)}
\newcommand{\ve}{\varepsilon}
\newcommand{\es}{\emptyset}

\newcommand{\sst}{\subset}
\newcommand{\sse}{\subseteq}
\newcommand{\spt}{\supset}
\newcommand{\spe}{\supseteq}

\newcommand{\ie}{\text{ i.e., }}
\newcommand{\st}{\text{ s.t.  }}

% Declare delimiter
\DeclareMathDelimiter{\lVert}
  {\mathopen}{symbols}{"6B}{largesymbols}{"0D}
\DeclareMathDelimiter{\rVert}
  {\mathclose}{symbols}{"6B}{largesymbols}{"0D}
\DeclarePairedDelimiter\norm{\lVert}{\rVert}
\DeclarePairedDelimiter\vct{\langle}{\rangle}

%%%%%%%%%%%%%%%%%%%%%%%%%%%%%%%%%%%%%%%%%%%%%%%%%%%

\theoremstyle{plain}
\newtheorem{corollary}{Corollary}
\newtheorem{proposition}{Proposition}

%%%%%%%%%%%%%%%%%%%%%%%%%%%%%%%%%%%%%%%%%%%%%%%%%%%

\setCJKmainfont{思源宋體 TW}
%\setmainfont{思源宋體 TW}

%%%%%%%%%%%%%%%%%%%%%%%%%%%%%%%%%%%%%%%%%%%%%%%%%%%

% Use listing package to display different programming language
% Use xcolor package to display syntax color for programming language
% Usage: \env{lstlisting}[language=LANGUAGENAME]
% Common language name: Python, bash, C, C++, R, sh, make, Matlab

\usepackage{listings}
\usepackage{xcolor}

\definecolor{dkgreen}{rgb}{0,0.6,0}
\definecolor{gray}{rgb}{0.5,0.5,0.5}
\definecolor{mauve}{rgb}{0.58,0,0.82}
\lstset{frame=tb,
  language=Python,
      aboveskip=3mm,
    belowskip=3mm,
    showstringspaces=false,
  columns=flexible,
      basicstyle={\small\ttfamily},
    numbers=none,
    numberstyle=\tiny\color{gray},
  keywordstyle=\color{blue},
      commentstyle=\color{dkgreen},
    stringstyle=\color{mauve},
    breaklines=true,
  breakatwhitespace=true,
      tabsize=3
  }

%%%%%%%%%%%%%%%%%%%%%%%%%%%%%%%%%%%%%%%%%%%%%%%%%%%

\begin{document}
%\begin{CJK*}{UTF8}{bsmi}
\course{Introduction to Mathematical Analysis (I)}
\semester{Fall 2021}
%\teacher{}
\department{}
\assignment{6}
\date{\today}
\student[$\dagger$]{Yu-Chieh Kuo}{B07611039}
\affil[$\dagger$]{Department of Information Management, National Taiwan University}
\ujtitle

% p1
\problem{}
To prove that $C+C = [0,2]$, we show that both directiond are correct.
\begin{itemize}
    \item {\bf $(C+C \sse [0,2])$} \\
        If $x \in C$, let $x = \sum^{\infty}_{n=1}\frac{a_n}{3^n}$, where $a_n \in \{0,2\}$.
        Therefore, any element of $C+C$ is of the form
        \[
            \sum^{\infty}_{n=1}\frac{a_n}{3^n} +  \sum^{\infty}_{n=1}\frac{b_n}{3_n}
            = \sum^{\infty}_{n=1}\frac{a_n+b_n}{3^n}
            = 2\sum^{\infty}_{n=1}\frac{(a_n+b_n)/2}{3^n}
            = 2\sum^{\infty}_{n=1}\frac{x_n}{3^n},
        \]
        here $b_n = 0,2$ as well as $a_n$, and $x_n = \frac{a_n+b_n}{2}$ for $n \in \N$.
        It's clear to show that $x_n \in \{0,1,2\}$ as $a_n,b_n \in \{0,2\}$, $n \in \N$.
        Then, it suffices to say 
        \[
            \sum^{\infty}_{n=1}\frac{x_n}{3^n} \in [0,1] \implies
            2\sum^{\infty}_{n=1}\frac{x_n}{3^n} \in [0,2].
        \]
        
    \item {\bf $([0,2] \sse C+C)$} \\
        Note that $[0,2] \sse C+C$ is equivalent to $[0,1] \sse \frac{C}{2} + \frac{C}{2}$ here,
        and we show the latter form as below. The reason I prove $[0,1] \sse \frac{C}{2} + \frac{C}{2}$
        to show $[0,2] \sse C+C$ is that $[0,1]$ is more closed to the form we derive in the first
        direction. Futhermore, we could easily put $x \in [0,1]$ and $x = \sum^{\infty}_{n=1}\frac{x_n}{3^n}$,
        where $x_n \in \{0,1,2\}$.
        Next, we need to select two element $y,z \in \frac{C}{2}$ such that $x = y+z$.
        Define $y = \sum^{\infty}_{n=1}\frac{y_n}{3^n}$ and $z = \sum^{\infty}_{n=1}\frac{z_n}{3^n}$,
        for all $n \in \N$, $y_n = 0$ if $x_n = 0$ and $y_n = 1$ if $x_n = 1,2$;
        for all $n \in \N$, $z_n = 0$ if $x_n = 0,1$ and $z_n = 1$ if $x_n = 2$.
        Such $x,y,z$ follow $y,z \in \frac{C}{2}$ and $x=y+z$, that is,
        \[
            x = y+z \in \frac{C}{2} + \frac{C}{2} \implies [0,1] \sse \frac{C}{2} + \frac{C}{2}.
        \]

\end{itemize}

% p2
\problem{}
\begin{enumerate}[label = (\alph*)]
    \item If every porper subsequence of $\{x_n\}$ converges, then for the origin sequence $\{x_n\}$, for arbitrary
        integer $N$, the subsequence $\{x_1,x_2,\dots,x_{N_1},x_{N+1},\dots\}$ converges as well.
        Hence, the origin sequence $\{x_n\}$ converges.

    \item It's the fact that every Cauchy sequence is bounded, and for $\{x_n\}$ is monotonic,
        $\{x_n\}$ is convergent.

    \item Suppose the subsequence of $\{x_n\}$ converges to $x$. Let $\ve > 0$, for some large enough
        $N \in \N$, 
        \[
            d(x_m, x_n) < \frac{\ve}{2}, \quad \text{if } m, n>N. 
        \]
        Similarly, for some large enough $N' \in \N$, and $N' > N$ such that for $k > N'$, 
        $d(x_{n_k},x) < \frac{\ve}{2}$. Then, for $n>N'$, we have
        \[
            d(x_n,x) \leq d(x_n,x_{n_{N'}}) + d(x_{n_{N'}},x) < \ve.
        \]
        Hence $\{x_n\}$ converges to $x$.

    \item Assume the statement is false, then there exists $\ve > 0$ such that for every $k$, there exists
        $n_k > k$ satisfying $d(x_{n_k},p)\geq \ve$. This implies that no any subsequence converges to $p$,
        contrary to the assumption. As a consequence, the statement implies the convergence of $\{x_n\}$.

    \item It's similar to 2.(d) and hence the statement implies the convergence of $\{x_n\}$.
\end{enumerate}

% p3
\problem{}
\begin{enumerate}[label = (\alph*)]
    \item If $\{a_n\}$ is unbounded, then $a_n \to \infty$ as $n \to \infty$. Also,
        \[
            \lim_{n\to\infty}\frac{a_n}{1+a_n} = \lim_{n\to\infty}\frac{1}{\frac{1}{a_n}+1} = 1.
        \]
        As $\frac{a_n}{1+a_n}$ does not converge to $0$ when $n \to \infty$, by {\bf Theorem 3.23} in Rudin, we claim
        that $\frac{a_n}{1+a_n}$ diverges.

        If $\{a_n\}$ is bounded, then there exists $M \in \R$ such that for all $n\in\N$, $a_n \leq M$.
        It shows that $\frac{a_n}{1+a_n} \geq \frac{1}{1+M}$, indicating $\frac{a_n}{1+a_n}$  diverges as well.
        
    \item Since for all $n \in \N$ $a_n > 0 \ie s_{N+i} \geq s_{N+j}$ for $i\geq j$, $i,j \in \N$, we have
        \[
            \begin{array}{RCL}
                \frac{a_{N+1}}{s_{N+1}} + \cdots + \frac{a_{N+k}}{s_{N+k}} & \geq & 
                \frac{a_{N+1}}{s_{N+k}} + \cdots + \frac{a_{N+k}}{s_{N+k}} \\ [3mm]
                & = & \frac{1}{s_{N+k}}(a_{N+1} + \cdots + a_{N+k}) \\ [3mm]
                & = & \frac{s_{N+k}-s_N}{s_{N+k}} \\ [3mm]
                & = & 1 - \frac{s_N}{s_{N+k}}.
            \end{array}
        \]
        It shows that the partial sum of $\sum \frac{a_n}{s_n}$ does not in the form of Cauchy sequence,
        then for every $N \in \N$, we could pick any large enough $k\in\N$ such that 
        $1-\frac{s_N}{s_{N+k}} \geq \frac{1}{2}$, deducing the fact that $\sum\frac{a_n}{s_n}$ diverges.

    \item Since for all $n \in \N$ $a_n > 0 \ie s_{N+i} \geq s_{N+j}$ for $i\geq j$, $i,j \in \N$, we have
        \[
            \frac{1}{s_n-1} - \frac{1}{s_n} = \frac{s_n - s_{n-1}}{s_{n-1}\cdot s_n} \geq \frac{a_n}{s_n^2}.
        \]
        Futhermore, $\sum\frac{1}{s_{n-1}} - \frac{1}{s_n}$ converging to $\frac{1}{a_1}$ implies that, 
        by {\bf Theorem 3.25} in Rudin, $\sum\frac{a_n}{s_n^2}$ converges.

    \item To check the convergence of $\sum\frac{a_n}{1+na_n}$, consider two cases as below.
        \begin{enumerate}
            \item Let $a_n = \frac{1}{n}$ and hence $\frac{a_n}{1+na_n} = \frac{1}{2n}$. It shows that
                $\sum\frac{a_n}{1+na_n} = \frac{1}{2}\sum\frac{1}{n}$ diverges.

            \item Define $a_n$ as
                \[
                    a_n = \begin{cases}
                        2^k & \text{if } n=2^k \ \forall k = 0,1,2,\dots \\
                        0 & \text{otherwose.}
                    \end{cases}
                \]
                Therefore,
                \[
                    \sum\frac{a_n}{1+na_n} = \sum\frac{2^k}{1+2^{2k}} < \sum\frac{1}{2^k},
                \]
                and $\sum\frac{1}{2^k}$ diverges by {\bf Theorem 3.26} in Rudin, it indicates that
                $\sum\frac{a_n}{1+na_n}$ converges.
        \end{enumerate}
        As a result, $\sum\frac{a_n}{1+na_n}$ can either converge or diverge, depending on $\{a_n\}$.

        Move on the discussion of $\sum\frac{a_n}{1+n^2a_n}$. Since $a_n > 0$, we have
        $n^2a_n < 1+n^2a_n$ and therefore
        \[
            \sum\frac{a_n}{1+n^2a_n} < \sum\frac{1}{n^2}.
        \]
        By {\bf Theorem 3.28} in Rudin, $\sum\frac{a_n}{1+n^2a_n}$ converges.
\end{enumerate}

% p4
\problem{}
Define $x_n = \frac{s_n}{n+1}$ to ease note. Here I want to start the derivation from prooving
$\limsup_{n\to\infty}x_n \leq \limsup_{n\to\infty}a_n$.
If $\limsup_{n\to\infty}a_n = +\infty$, it's nothing to do. Next, if $\limsup_{n\to\infty}a_n = a$
where $a \in \R$, for any $\ve > 0$, there exists a $N \in \Z_+$ such that as $n \geq N$,
$a_n < a + \ve$. Therefore, we have
\[
    \begin{array}{RCL}
        x_n & = & \frac{a_1 + \cdots + a_N}{n+1} + \frac{a_{N+1}+\cdots + a_n}{n+1} \\
        & \leq & \frac{a_1 + \cdots + a_N}{n+1} + (a+\ve)\frac{n-N}{n+1},
    \end{array}
\]
implying that $\limsup_{n\to\infty}x_n \leq a+\ve \implies \limsup_{n\to\infty}x_n \leq a$.

Finally, if $\limsup_{n\to\infty}a_n = -\infty$, it implies $\lim_{n\to\infty}a_n = -\infty$.
For any $M > 0$, $M \in \R$, there exists a $N \in \Z_+$ such that as $n \geq N$,
$a_n \leq -M$. Therefore, we have
\[
    \begin{array}{RCL}
        x_n & = & \frac{a_1 + \cdots + a_N}{n+1} + \frac{a_{N+1} + \cdots + a_n}{n+1} \\
        & \leq & \frac{a_1 + \cdots + a_N}{n+1} - M \frac{n-N}{n+1},
    \end{array}
\]
implying that $\limsup_{n\to\infty}x_n \leq -M \implies \limsup_{n\to\infty}x_n = -\infty$.

Based on the cases we discuss above, it has been proved that $\limsup_{n\to\infty}x_n \leq 
\limsup_{n\to\infty}a_n$. Similarly to show that $\liminf_{n\to\infty}a_n \leq
\liminf_{n\to\infty}x_n$.
At the end, 
\[
    \liminf_{n\to\infty}a_n \leq \liminf_{n\to\infty}x_n \leq \limsup_{n\to\infty}x_n \leq
    \limsup_{n\to\infty}a_n.
\]

If $a_n \to l$ as $n \to \infty$, $\liminf_{n\to\infty}a_n = \limsup_{n\to\infty}a_n = l$,
and at that case $\liminf_{n\to\infty}\frac{s_n}{n+1} = \limsup_{n\to\infty}\frac{s_n}{n+1}
= l$. The converse direction does not hold, and 
$a_n = (-1)^n$ is a case. It shows that 
\[
    \liminf_{n\to\infty}\frac{s_n}{n+1} = \limsup_{n\to\infty}\frac{s_n}{n+1} = 0.
\]
However, $a_n$ diverges.

% p5
\problem{}
\begin{enumerate}[label = (\alph*)]
    \item This problem is not required to do.
        
    \item Since 
        \[
            (1+\frac{1}{n})cosn\pi = \begin{cases}
                1 & \text{if } n = 2k \\
                -1 & \text{if } n = 2k+1,
            \end{cases}
        \]
        it's easy to show that
        \[
            \limsup_{n\to\infty}(1+\frac{1}{n})cosn\pi = 1,
        \]
        \[
            \liminf_{n\to\infty}(1+\frac{1}{n})cosn\pi = -1.

        \]

    \item Note that
        \[
            nsin(\frac{n\pi}{3}) = \begin{cases}
                (1+6k)sin\left(\frac{\pi}{3}\right) & \text{if } n = 1 + 6k \\
                -(4+6k)sin\left(\frac{\pi}{3}\right) & \text{if } n = 4 + 6k.
            \end{cases}
        \]
        Therefore, we have
        \[
            \limsup_{n\to\infty}nsin\left(\frac{n\pi}{3}\right) = \infty,
        \]
        \[
            \liminf_{n\to\infty}nsin\left(\frac{n\pi}{3}\right) = -\infty.
        \]

    \item Since $sin\left(\frac{n\pi}{2}\right)cos\left(\frac{n\pi}{2}\right)
        = \frac{sin(n\pi)}{2} = 0$, it's easy to say that
        \[
            \limsup_{n\to\infty}sin\left(\frac{n\pi}{2}\right)
            cos\left(\frac{n\pi}{2}\right)=
            \liminf_{n\to\infty}sin\left(\frac{n\pi}{2}\right)
            cos\left(\frac{n\pi}{2}\right)= 0.
        \]

    \item Since $\lim_{n\to\infty}\frac{(-1)^n n}{(1+n)^n} = 0$,
        therefore,
        \[
            \limsup_{n\to\infty}\frac{(-1)^n n}{(1+n)^n} = 
            \liminf_{n\to\infty}\frac{(-1)^n n}{(1+n)^n} = 0.
        \]

    \item Note that 
        \[
            \frac{n}{3} - \left\lceil\frac{n}{3}\right\rceil = \begin{cases}
                \frac{1}{3} & \text{if } n = 3k+1 \\
                \frac{2}{3} & \text{if } n = 3k+2 \\
                0 & \text{if } n = 3k,
            \end{cases}
        \]
        where $k = 0,1,2,\dots$. Therefore, it's easy to say that
        \[
            \limsup_{n\to\infty}\frac{n}{3} - \left\lceil\frac{n}{3}\right\rceil = \frac{2}{3},
        \]
        \[
            \liminf_{n\to\infty}\frac{n}{3} - \left\lceil\frac{n}{3}\right\rceil = 0.
        \]
\end{enumerate}

% p6
\problem{}
\begin{enumerate}[label = (\alph*)]
    \item (Not required) 
    \item (Not required)
    \item (Not required)

    \item Note that $\frac{1}{n^p - n^q} = \frac{1}{n} \cdot \frac{1}{1-n^{q-p}}$,
        for the case $p>1$, by Limit Comparison Test with $\frac{1}{n^p}$, we have
        \[
            \lim_{n\to\infty}\frac{\frac{1}{n^p-n^q}}{\frac{1}{n^p}} = 1,
        \]
        which means the series converges.
        
        For the case $p\leq1$, also, by Limit Comparison Test with $\frac{1}{n^p}$, we have
        \[
            \lim_{n\to\infty}\frac{\frac{1}{n^p-n^q}}{\frac{1}{n^p}} = 1,
        \]
        which means the series diverges.

    \item Since $n^{-1-\frac{1}{n}} \geq n^{-1}$ for all $n\in\R$, the series diverges.

    \item Note that $\frac{1}{n^p - n^q} = \frac{1}{n} \cdot \frac{1}{1-n^{q-p}}$,
        for the case $p>1$, by Limit Comparison Test with $\frac{1}{p^n}$, we have
        \[
            \lim_{n\to\infty}\frac{\frac{1}{p^n-q^n}}{\frac{1}{p^n}} = 1,
        \]
        and the series converges.
        For the case $p\leq1$, also, by Limit Comparison Test with $\frac{1}{p^n}$, we have
        \[
            \lim_{n\to\infty}\frac{\frac{1}{p^n-q^n}}{\frac{1}{p^n}} = 1,
        \]
        and this means the series diverges.

    \item Since $\lim_{n\to\infty}\frac{1}{n\ln\left(1+\frac{1}{n}\right)} = 1$, the series 
        diverges.

    \item Note that $\log n^{\log n} = n^{\log\log n}$, the series goes to
        \[
            \sum^{\infty}_{n=2}\frac{1}{n^{\log\log n}},
        \]
        and for some large enough $N \in \N$ such that if $n > N$, then $\log\log n > 1$.
        Hence, the series converges.

    \item By dividing $p$ into three cases here, we could start the proof.
        \begin{enumerate}
            \item ($p\leq0$): \\
                We derive that
                \[
                    \frac{1}{n\log n(\log\log n)} \geq \frac{1}{n\log n}
                \]
                for $n\geq 3$, due to the divergence of $\sum\frac{1}{n\log n}$,
                it's easy to state that the series diverges.
           
            \item ($0<p<1$): \\
                Pick a large enough $N \in \N$ and by Cauchy Condensation Test, we have
                \[
                    \sum^\infty_{k=N}\frac{2^k}{2^k\log 2^k(\log\log 2^k)^p}
                    = \frac{1}{\log 2}\sum^\infty_{k=N}\frac{1}{i(\log k \log 2)^p}
                    \geq \sum^\infty_{k=N}\frac{1}{k(\log k)^p}.
                \]
                To the convergence of $\sum\frac{1}{k(\log k)^p}$, by applying Cauchy Condensation Test,
                it shows
                \[
                    \sum\frac{1}{k^p(\log 2)^p} > \sum\frac{1}{k^p}
                \]
                which means $\sum\frac{1}{k(\log k)^p}$ diverges. Hence, the origin series diverges
                as well.

            \item ($p \geq 1$): \\
                Pick a large enough $N \in \N$ and by Cauchy Condensation Test, we have
                \[
                    \begin{array}{RCL}
                        \sum^\infty_{k=N}\frac{2^k}{2^k\log 2^k(\log\log 2^k)^p} 
                        & = & \frac{1}{\log 2}\sum^\infty_{k=N}\frac{1}{k(\log k \log 2)^p} \\
                        & \leq & 2\sum^\infty_{k=N}\frac{1}{k(\log k \log 2)^p} \\
                        & \leq & 4\sum^\infty_{k=N}\frac{1}{k(\log k)^p}.
                    \end{array}
                \]
                The convergence of $\sum\frac{1}{k(\log k)^p}$ has been discussed above.
                Consequently, we could say that the origin series diverges since
                $\sum\frac{1}{k(\log k)^p}$ diverges.

        \end{enumerate}

    \item Consider $a_n = \left(\frac{1}{\log\log n}\right)^{\log\log n}$, 
        $b_n = \frac{1}{n}$ for $n\geq 3$ and construct $n = e^{e^x}$ for the future use. Then,
        by Limit Comparison Test, we have
        \[
            \frac{a_n}{b_n} = \frac{n}{\left(\log\log n\right)^{\log\log n}} = \frac{e^{e^x}}{x^x}.
        \]
        As $n\to\infty$, $\frac{e^{e^x}}{x^x} \to \infty$ as well. Therefore, the series diverges.
\end{enumerate}

% p7
\problem{}
\begin{enumerate}[label = (\alph*)]
    \item Since $\sum\frac{1}{n}$ diverges, by Limit Comparison Test, we could clearly
        state that the series $\sum a_n$ diverges.

    \item The existence of $\lim(n^2a_n)$ indicates that the series $a_n$ is bounded,
        therefore, there exists some $M > 0, M \in N$ such that
        \[
            0 < n^2a_n \leq M \iff 0 < a_n \leq \frac{M}{n^2}.
        \]
        Since $\sum\frac{M}{n^2}$ converges, by comparison test, $\sum a_n$ converges.
\end{enumerate}

%\end{CJK*}
\end{document}

